\documentclass[12pt]{article}
% Page setup
\usepackage[margin=1in]{geometry}
\usepackage{parskip}
\usepackage{titling}
\setlength{\droptitle}{-6em} 

% Math packages
\usepackage{amsmath, amsthm, amssymb, amsfonts}
\usepackage{mathtools}
\makeatletter
\g@addto@macro\th@plain{\normalfont} % remove italics from theorems (plain style)
\makeatother

% Commutative diagrams and proofs
\usepackage{tikz-cd}
\usepackage{proof}


% Theorem environments
\newtheorem{theorem}{Theorem}[section]
\newtheorem{lemma}[theorem]{Lemma}
\newtheorem{proposition}[theorem]{Proposition}
\newtheorem{corollary}[theorem]{Corollary}
\newtheorem{fact}[theorem]{Fact}
\newtheorem{remark}[theorem]{Remark}

\theoremstyle{definition}
\newtheorem{definition}[theorem]{Definition}
\newtheorem{example}[theorem]{Example}

% Useful commands for category theory
\newcommand{\mc}{\mathcal}
\newcommand{\ini}{\operatorname{ini}}
\newcommand{\ter}{\operatorname{ter}}
\newcommand{\cod}{\operatorname{cod}}
\newcommand{\dom}{\operatorname{dom}}

\newcommand{\List}{\operatorname{List}}
\newcommand{\lop}{\operatorname{[ \ ]}}
\newcommand{\len}{\operatorname{len}}
\newcommand{\type}{\operatorname{type}}

\newcommand{\Exc}{\operatorname{Exception}}

\newcommand{\Kl}{\operatorname{Kl}}
\newcommand{\plus}{\operatorname{+}}

\newcommand{\Cat}{\mathcal{C}at}
\newcommand{\Vect}{\mathrm{Vect}_\mathbf{k}}
\newcommand{\Hom}{\operatorname{Hom}}
\newcommand{\Nat}{\operatorname{Nat}}
\newcommand{\id}{\mathrm{id}}
\newcommand{\op}{\mathrm{op}}
\newcommand{\obj}{\operatorname{obj}}
\newcommand{\arr}{\operatorname{arr}}
\newcommand{\adj}{\operatorname{adj}}
\newcommand{\coadj}{\operatorname{coadj}}
\newcommand{\colim}{\operatorname*{colim}}
\newcommand{\lcoim}{\operatorname*{(co)lim}}

% Arrows
\newcommand{\ra}{\rightarrow}
\newcommand{\from}{\leftarrow}
\newcommand{\To}{\Rightarrow}
\newcommand{\xto}{\xrightarrow}
\newcommand{\xfrom}{\xleftarrow}

\renewcommand{\labelitemi}{--} % Level 1: Dashes
\renewcommand{\labelitemii}{$\circ$} % Level 2: Circles
\renewcommand{\labelitemiii}{$\bullet$} % Level 3: Bullets (default)


\title{(Co)Limits}
\author{Alyson Mei}
\date{\today}


\begin{document}

\maketitle

\tableofcontents

\newpage

\section{(Co)Limits}

\subsection{Main definitions}

\begin{definition}[Comma category]
Given functors

$$\mc{A} \xto{S} \mc{C} \xfrom{T} \mc{B},$$

the category $(S \downarrow T)$ is defined as follows:

\begin{itemize}
\item Objects of are the triples $(a: \obj(\mc{A}), h:\arr(\mc{C}), b: \obj(\mc{B})),$

\item Morphisms are pairs $(f: \arr(\mc{A}), g:\arr(\mc{B}))$ satisfying the condition:
% https://q.uiver.app/#q=WzAsNixbMCwwLCIoYSwgaCwgYikiXSxbMCwxLCIoYScsIGgnLCBiJykiXSxbMSwwLCJTYSJdLFsxLDEsIlNhJyJdLFsyLDAsIlRiIl0sWzIsMSwiVGInIl0sWzAsMSwiKGYsIGcpIl0sWzIsNCwiaCJdLFszLDUsImgnIiwyXSxbMiwzLCJTZiIsMl0sWzQsNSwiVGYiXV0=
\[\begin{tikzcd}
	{(a, h, b)} & Sa & Tb \\
	{(a', h', b')} & {Sa'} & {Tb'}
	\arrow["{(f, g)}", from=1-1, to=2-1]
	\arrow["h", from=1-2, to=1-3]
	\arrow["Sf"', from=1-2, to=2-2]
	\arrow["Tf", from=1-3, to=2-3]
	\arrow["{h'}"', from=2-2, to=2-3]
\end{tikzcd}
\text{.}
\]
\end{itemize}
\end{definition}

The concept of a comma category allows us to define (co)limits in a uniform way.
\begin{definition}[(Co)limit]
The limit of a diagram $D$ is defined as a terminal object of the category $(\Delta \downarrow D)$. Dually, the colimit of a diagram $E$ is defined as an initial object of the category $(E \downarrow \Delta)$.
\[
\mc{D} \xrightarrow{\Delta} \mc{D}^{\mc{J}} \xleftarrow{D} 1;
\qquad
1 \xrightarrow{E} \mc{C}^{\mc{J}} \xleftarrow{\Delta} \mc{C}.
\]
Notation:
\[
\lim_{\mc{J}} D := \ter(\Delta \downarrow D);
\qquad
\colim_{\mc{J}} E := \ini(E \downarrow \Delta).
\]
Visual intuition:
% https://q.uiver.app/#q=WzAsOCxbNCwyLCJEal8xIl0sWzcsMiwiRGpfMiJdLFs1LDMsIkRqX24iXSxbNywwLCJcXGxpbSBEIl0sWzQsMCwiZCJdLFswLDAsIlxcRGVsdGEgX2QiXSxbMiwyLCJEIl0sWzIsMCwiXFxEZWx0YV97XFxsaW0gRCB9Il0sWzAsMSwiRGciLDEseyJsYWJlbF9wb3NpdGlvbiI6MzAsInN0eWxlIjp7ImJvZHkiOnsibmFtZSI6ImRvdHRlZCJ9fX1dLFsxLDIsIkRmIiwxXSxbMCwyLCJEZlxcY2lyYyBEZyIsMl0sWzMsMCwiaF8xIiwxLHsibGFiZWxfcG9zaXRpb24iOjcwLCJzdHlsZSI6eyJib2R5Ijp7Im5hbWUiOiJkb3R0ZWQifX19XSxbMywxLCJoXzIiLDEseyJsYWJlbF9wb3NpdGlvbiI6NzB9XSxbMywyLCJoX24iLDEseyJsYWJlbF9wb3NpdGlvbiI6ODB9XSxbNCwzLCIiLDAseyJzdHlsZSI6eyJib2R5Ijp7Im5hbWUiOiJkYXNoZWQifX19XSxbNCwwXSxbNCwyXSxbNCwxLCIiLDAseyJzdHlsZSI6eyJib2R5Ijp7Im5hbWUiOiJkb3R0ZWQifX19XSxbNSw3LCIiLDAseyJzdHlsZSI6eyJib2R5Ijp7Im5hbWUiOiJkYXNoZWQifX19XSxbNyw2LCJoIl0sWzUsNiwiaCciLDJdXQ==
\[\begin{tikzcd}
	{\Delta _d} && {\Delta_{\lim D }} && d &&& {\lim D} \\
	\\
	&& D && {Dj_1} &&& {Dj_2} \\
	&&&&& {Dj_n}
	\arrow[dashed, from=1-1, to=1-3]
	\arrow["{h'}"', from=1-1, to=3-3]
	\arrow["h", from=1-3, to=3-3]
	\arrow[dashed, from=1-5, to=1-8]
	\arrow[from=1-5, to=3-5]
	\arrow[dotted, from=1-5, to=3-8]
	\arrow[from=1-5, to=4-6]
	\arrow["{h_1}"{description, pos=0.7}, dotted, from=1-8, to=3-5]
	\arrow["{h_2}"{description, pos=0.7}, from=1-8, to=3-8]
	\arrow["{h_n}"{description, pos=0.8}, from=1-8, to=4-6]
	\arrow["Dg"{description, pos=0.3}, dotted, from=3-5, to=3-8]
	\arrow["{Df\circ Dg}"', from=3-5, to=4-6]
	\arrow["Df"{description}, from=3-8, to=4-6]
\end{tikzcd}
\text{.}\]

\end{definition}

\begin{remark}
    From here on we use $\lcoim D$ also to denote colimit object. 
\end{remark}

\begin{definition}[(Co)limit funtor]
    
If $\mc{C}$ has all (co)limits of shape $\mc{J}$, then there exists a (co)limit functor
$$\lcoim_{\mc{J}} \colon \mc{C}^{\mc{J}} \to \mc{C}.$$
The action on morphisms is defined via the universal property:
% https://q.uiver.app/#q=WzAsNixbMiwwLCJEIl0sWzIsMiwiRCciXSxbMCwwLCJcXERlbHRhIFxcbGltIEQgIl0sWzAsMiwiXFxEZWx0YSBcXGxpbSBEJyAiXSxbNCwwLCJcXGxpbSBEIl0sWzQsMiwiXFxsaW0gRCcgIl0sWzIsMywiXFxEZWx0YShcXHBoaShcXGFscGhhIFxcY2lyYyBoKSkiLDIseyJzdHlsZSI6eyJib2R5Ijp7Im5hbWUiOiJkYXNoZWQifX19XSxbMiwwLCJoIl0sWzMsMSwiaCciLDJdLFswLDEsIlxcYWxwaGEiXSxbMiwxLCJcXGFscGhhIFxcY2lyYyBoIl0sWzQsNSwiXFxwaGkoXFxhbHBoYSBcXGNpcmMgaCkiLDAseyJzdHlsZSI6eyJib2R5Ijp7Im5hbWUiOiJkYXNoZWQifX19XV0=
\[\begin{tikzcd}
	{\Delta \lim D } && D && {\lim D} \\
	\\
	{\Delta \lim D' } && {D'} && {\lim D' }
	\arrow["h", from=1-1, to=1-3]
	\arrow["{\Delta(\phi(\alpha \circ h))}"', dashed, from=1-1, to=3-1]
	\arrow["{\alpha \circ h}", from=1-1, to=3-3]
	\arrow["\alpha", from=1-3, to=3-3]
	\arrow["{\phi(\alpha \circ h)}", dashed, from=1-5, to=3-5]
	\arrow["{h'}"', from=3-1, to=3-3]
\end{tikzcd}
\text{.}
\]

\end{definition}

\begin{remark}
    Thus, limit functor is a right adjoint to diagonal functor, colimit functor is a left adjoint to diagonal functor.
\end{remark}

\subsection{Examples}

\begin{remark}[Terminal and initial objects]
    Terminal and initial objects in a category are special cases of limits and colimits, respectively.
\end{remark}

\begin{definition}[(Co)products]
(Co)Product is defined as the (co)limit of a discrete diagram:
$$\prod_\mc{J} D := \lim_\mc{J} D; \qquad \coprod_\mc{J} E:= \colim_\mc{J} E.$$

Setting $J := \obj(\mc{J})$, one can write the standard low-level universal property diagrams for (co)product:
% https://q.uiver.app/#q=WzAsNixbMSwwLCJcXHN1bV97aiBcXGluIEp9IGNfaiJdLFswLDAsImNfayJdLFsxLDEsImMiXSxbMywxLCJcXHByb2Rfe2ogXFxpbiBKfSBkX2oiXSxbMywwLCJkIl0sWzQsMSwiZF9pIl0sWzEsMCwiaV9rIl0sWzEsMiwiZl9rIiwyXSxbMiwwLCJcXHN1bV97aiBcXGluIEp9IGZfayIsMix7InN0eWxlIjp7ImJvZHkiOnsibmFtZSI6ImRhc2hlZCJ9fX1dLFs0LDUsImdfayJdLFs0LDMsIlxccHJvZF97aiBcXGluIEp9IGdfayIsMix7InN0eWxlIjp7ImJvZHkiOnsibmFtZSI6ImRhc2hlZCJ9fX1dLFszLDUsIlxccGlfaiIsMl1d
\[\begin{tikzcd}
	{c_k} & {\coprod_{j \in J} c_j} && d \\
	& c && {\prod_{j \in J} d_j} & {d_i}
	\arrow["{i_k}", from=1-1, to=1-2]
	\arrow["{f_k}"', from=1-1, to=2-2]
	\arrow["{\prod_{j \in J} g_k}"', dashed, from=1-4, to=2-4]
	\arrow["{g_k}", from=1-4, to=2-5]
	\arrow["{\coprod_{j \in J} f_k}"', dashed, from=2-2, to=1-2]
	\arrow["{\pi_j}"', from=2-4, to=2-5]
\end{tikzcd}
\text{.}
\]
\end{definition}

\begin{definition}[(Co)Equalizers]
A (co)equalizer is defined as the (co)limit of a two-arrow diagram:
% https://q.uiver.app/#q=WzAsMyxbMSwwLCJqXzEiXSxbMiwwLCJqXzIiXSxbMCwwLCJcXG1je0p9OiJdLFswLDEsImZfMSIsMCx7Im9mZnNldCI6LTJ9XSxbMCwxLCJmXzIiLDIseyJvZmZzZXQiOjJ9XV0=
\[\begin{tikzcd}
	{\mc{J}:} & {j_1} & {j_2}
	\arrow["{f_1}", shift left=2, from=1-2, to=1-3]
	\arrow["{f_2}"', shift right=2, from=1-2, to=1-3]
\end{tikzcd}
\text{,}
\]
% https://q.uiver.app/#q=WzAsOCxbMCwwLCJFIl0sWzEsMCwiWCJdLFsyLDAsIlkiXSxbMCwxLCJPIl0sWzUsMCwiQyJdLFs0LDAsIlkiXSxbMywwLCJYIl0sWzUsMSwiUCJdLFsxLDIsImYiLDAseyJvZmZzZXQiOi0yfV0sWzEsMiwiZyIsMix7Im9mZnNldCI6Mn1dLFswLDEsIlxcdGV4dHJte2VxfSJdLFszLDAsIiIsMCx7InN0eWxlIjp7ImJvZHkiOnsibmFtZSI6ImRhc2hlZCJ9fX1dLFszLDEsImgnIiwyXSxbNiw1LCJmIiwwLHsib2Zmc2V0IjotMn1dLFs2LDUsImciLDIseyJvZmZzZXQiOjJ9XSxbNSw0LCJcXHRleHRybXtjb2VxfSJdLFs1LDcsImgnIiwyXSxbNCw3LCIiLDEseyJzdHlsZSI6eyJib2R5Ijp7Im5hbWUiOiJkYXNoZWQifX19XV0=
\[\begin{tikzcd}
	E & X & Y & X & Y & C \\
	O &&&&& P
	\arrow["{\textrm{eq}}", from=1-1, to=1-2]
	\arrow["f", shift left=2, from=1-2, to=1-3]
	\arrow["g"', shift right=2, from=1-2, to=1-3]
	\arrow["f", shift left=2, from=1-4, to=1-5]
	\arrow["g"', shift right=2, from=1-4, to=1-5]
	\arrow["{\textrm{coeq}}", from=1-5, to=1-6]
	\arrow["{h'}"', from=1-5, to=2-6]
	\arrow[dashed, from=1-6, to=2-6]
	\arrow[dashed, from=2-1, to=1-1]
	\arrow["{h'}"', from=2-1, to=1-2]
\end{tikzcd}
\text{.}\]
\end{definition}
\begin{remark}[(Co)Kernels]
    A (co)kernel is a special case of a (co)equalizer where one of the morphisms is a zero morphism.
\end{remark}

\begin{definition}[Pullbacks and pushouts]
    Pullback and pushout are defined as limit and colimit of the following diagrams:
    % https://q.uiver.app/#q=WzAsNixbMSwwLCJqXzMiXSxbMSwxLCJqXzEiXSxbMCwxLCJqXzIiXSxbMywwLCJqXzEiXSxbNCwwLCJqXzMiXSxbMywxLCJqXzIiXSxbMCwxLCJmXzIiXSxbMiwxLCJmXzEiLDJdLFszLDQsImZfMiJdLFszLDUsImZfMSIsMl1d
    \[\begin{tikzcd}
    	& {j_3} && {j_1} & {j_3} \\
    	{j_2} & {j_1} && {j_2}
    	\arrow["{f_2}", from=1-2, to=2-2]
    	\arrow["{f_2}", from=1-4, to=1-5]
    	\arrow["{f_1}"', from=1-4, to=2-4]
    	\arrow["{f_1}"', from=2-1, to=2-2]
    \end{tikzcd}\text{,}\]
    % https://q.uiver.app/#q=WzAsMTAsWzEsMSwiQSBcXHRpbWVzX0MgQiJdLFsxLDIsIkIiXSxbMiwyLCJDIl0sWzIsMSwiQSJdLFswLDAsIkQiXSxbNCwwLCJDIl0sWzUsMCwiQiJdLFs1LDEsIkEgK19DIEIiXSxbNCwxLCJBIl0sWzYsMiwiRCJdLFsxLDJdLFszLDJdLFswLDFdLFswLDNdLFs0LDAsIiIsMix7InN0eWxlIjp7ImJvZHkiOnsibmFtZSI6ImRhc2hlZCJ9fX1dLFs0LDFdLFs0LDNdLFs1LDhdLFs1LDZdLFs4LDddLFs2LDddLFs3LDksIiIsMCx7InN0eWxlIjp7ImJvZHkiOnsibmFtZSI6ImRhc2hlZCJ9fX1dLFs4LDldLFs2LDldXQ==
    \[\begin{tikzcd}
    	D &&&& C & B \\
    	& {A \times_C B} & A && A & {A +_C B} \\
    	& B & C &&&& D
    	\arrow[dashed, from=1-1, to=2-2]
    	\arrow[from=1-1, to=2-3]
    	\arrow[from=1-1, to=3-2]
    	\arrow[from=1-5, to=1-6]
    	\arrow[from=1-5, to=2-5]
    	\arrow[from=1-6, to=2-6]
    	\arrow[from=1-6, to=3-7]
    	\arrow[from=2-2, to=2-3]
    	\arrow[from=2-2, to=3-2]
    	\arrow[from=2-3, to=3-3]
    	\arrow[from=2-5, to=2-6]
    	\arrow[from=2-5, to=3-7]
    	\arrow[dashed, from=2-6, to=3-7]
    	\arrow[from=3-2, to=3-3]
    \end{tikzcd}
    \text{.}
    \]
\end{definition}



\subsection{(Co)Continuous functors}



\begin{definition}[(Co)Continuous functors]
A functor $F: \mc{C} \to \mc{D}$ is called (co)continuous if it preserves (co)limits:
$$F \textrm{(co)lim}\ D \cong \textrm{(co)lim} \ F \circ D.$$
\end{definition}

\end{document}
