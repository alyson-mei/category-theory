\documentclass[12pt]{article}
% Page setup
\usepackage[margin=1in]{geometry}
\usepackage{parskip}
\usepackage{titling}
\setlength{\droptitle}{-6em} 

% Math packages
\usepackage{amsmath, amsthm, amssymb, amsfonts}
\usepackage{mathtools}
\makeatletter
\g@addto@macro\th@plain{\normalfont} % remove italics from theorems (plain style)
\makeatother

% Commutative diagrams and proofs
\usepackage{tikz-cd}
\usepackage{proof}


% Theorem environments
\newtheorem{theorem}{Theorem}[section]
\newtheorem{lemma}[theorem]{Lemma}
\newtheorem{proposition}[theorem]{Proposition}
\newtheorem{corollary}[theorem]{Corollary}
\newtheorem{fact}[theorem]{Fact}
\newtheorem{remark}[theorem]{Remark}

\theoremstyle{definition}
\newtheorem{definition}[theorem]{Definition}
\newtheorem{example}[theorem]{Example}

% Useful commands for category theory
\newcommand{\mc}{\mathcal}
\newcommand{\ini}{\operatorname{ini}}
\newcommand{\ter}{\operatorname{ter}}
\newcommand{\cod}{\operatorname{cod}}
\newcommand{\dom}{\operatorname{dom}}

\newcommand{\List}{\operatorname{List}}
\newcommand{\lop}{\operatorname{[ \ ]}}
\newcommand{\len}{\operatorname{len}}
\newcommand{\type}{\operatorname{type}}

\newcommand{\Exc}{\operatorname{Exception}}

\newcommand{\Kl}{\operatorname{Kl}}
\newcommand{\plus}{\operatorname{+}}

\newcommand{\Cat}{\mathcal{C}at}
\newcommand{\Vect}{\mathrm{Vect}_\mathbf{k}}
\newcommand{\Hom}{\operatorname{Hom}}
\newcommand{\Nat}{\operatorname{Nat}}
\newcommand{\id}{\mathrm{id}}
\newcommand{\op}{\mathrm{op}}
\newcommand{\obj}{\operatorname{obj}}
\newcommand{\arr}{\operatorname{arr}}
\newcommand{\adj}{\operatorname{adj}}
\newcommand{\coadj}{\operatorname{coadj}}
\newcommand{\colim}{\operatorname*{colim}}
\newcommand{\lcoim}{\operatorname*{(co)lim}}

% Arrows
\newcommand{\ra}{\rightarrow}
\newcommand{\from}{\leftarrow}
\newcommand{\To}{\Rightarrow}
\newcommand{\xto}{\xrightarrow}
\newcommand{\xfrom}{\xleftarrow}

\renewcommand{\labelitemi}{--} % Level 1: Dashes
\renewcommand{\labelitemii}{$\circ$} % Level 2: Circles
\renewcommand{\labelitemiii}{$\bullet$} % Level 3: Bullets (default)


\title{Comma category}
\author{Alyson Mei}
\date{\today}


\begin{document}

\maketitle

\tableofcontents

\newpage

\section{Diagonal functor}

\begin{definition}[Constant functor]
    Consider a category $\mc{C}$. A constant functor $$\Delta_c: \mc{D} \to \mc{C}$$ for a given $c:\obj({\mc{C}})$ is defined as $\lambda f. \id_c$:
    % https://q.uiver.app/#q=WzAsNCxbMCwwLCJkIl0sWzAsMSwiZCciXSxbMiwwLCJcXERlbHRhX2MoZCkgOj0gYyJdLFsyLDEsIlxcRGVsdGFfYyhkJykgOj0gYyJdLFswLDEsImYiLDJdLFsyLDMsIlxcRGVsdGFfYyAoZikgOj1cXGlkX2MiXV0=
    \[\begin{tikzcd}
    	d && {\Delta_c(d) := c} \\
    	{d'} && {\Delta_c(d') := c}
    	\arrow["f"', from=1-1, to=2-1]
    	\arrow["{\Delta_c (f) :=\id_c}", from=1-3, to=2-3]
    \end{tikzcd}
    \text{.}
    \]
\end{definition}

\begin{remark}[Natural transformations between constant functors]
    Natural transformation $\alpha^{(f)}: \Delta_c \to \Delta_{c'}$ between constant functors corresponds to morphisms
    between their values:
% https://q.uiver.app/#q=WzAsNixbMiwwLCJcXERlbHRhX2MoZCkgPSBjIl0sWzAsMCwiZCJdLFswLDEsImQnIl0sWzIsMSwiXFxEZWx0YV9jKGQnKSA9IGMiXSxbMywwLCJcXERlbHRhX3tjJ30oZCkgPSBjJyJdLFszLDEsIlxcRGVsdGFfe2MnfShkJykgPSBjJyJdLFsxLDIsImciLDJdLFswLDMsIlxcaWRfYyIsMl0sWzQsNSwiXFxpZF97Yyd9Il0sWzAsNCwiXFxhbHBoYV57KGYpfV9jID0gZiJdLFszLDUsIlxcYWxwaGFeeyhmKX1fe2MnfSA9IGYiLDJdXQ==
\[\begin{tikzcd}
	d && {\Delta_c(d) = c} & {\Delta_{c'}(d) = c'} \\
	{d'} && {\Delta_c(d') = c} & {\Delta_{c'}(d') = c'}
	\arrow["g"', from=1-1, to=2-1]
	\arrow["{\alpha^{(f)}_c = f}", from=1-3, to=1-4]
	\arrow["{\id_c}"', from=1-3, to=2-3]
	\arrow["{\id_{c'}}", from=1-4, to=2-4]
	\arrow["{\alpha^{(f)}_{c'} = f}"', from=2-3, to=2-4]
\end{tikzcd}\text{.}
\]
%In our notation will identify such natural transformations $\alpha^{(f)}$ with morphisms $f$. 
\end{remark}

\begin{definition}[Diagonal functor]
    Consider categories $\mc{C}, \mc{D}$. Diagonal functor $$\Delta: \mc{C} \to [\mc{D}, \mc{C}]$$ is defined as $\lambda f.\alpha^{(f)}$:
    % https://q.uiver.app/#q=WzAsNCxbMCwwLCJjIl0sWzAsMSwiYyciXSxbMSwwLCJcXERlbHRhKGMpIDo9IFxcRGVsdGFfYyJdLFsxLDEsIlxcRGVsdGEoYycpOj1cXERlbHRhX3tjJ30iXSxbMCwxLCJmIiwyXSxbMiwzLCJcXERlbHRhKGYpIDo9IFxcYWxwaGFeeyhmKX0iXV0=
    \[\begin{tikzcd}
    	c & {\Delta(c) := \Delta_c} \\
    	{c'} & {\Delta(c'):=\Delta_{c'}}
    	\arrow["f"', from=1-1, to=2-1]
    	\arrow["{\Delta(f) := \alpha^{(f)}}", from=1-2, to=2-2]
    \end{tikzcd}
    \text{.}
    \]
\end{definition}

\begin{remark}[Special case]
    Let $\mc{D}$ be small and discrete. Then $$[\mc{C}, \mc{D}] \simeq \prod_{d \in \obj(\mc{D})} \mc{C}$$
    via $F \mapsto (Fd)_{d \in \obj(\mc{D})}$. Thus, diagonal functor takes the following form:
    % https://q.uiver.app/#q=WzAsNCxbMCwwLCJjIl0sWzAsMSwiYyciXSxbMiwwLCJcXERlbHRhIChjKSA6PSBcXHByb2Rfe2QgXFxpbiBcXG9iaihcXG1je0R9KX0gYyJdLFsyLDEsIlxcRGVsdGEgKGMnKSA6PSBcXHByb2Rfe2QgXFxpbiBcXG9iaihcXG1je0R9KX0gYyciXSxbMCwxLCJmIiwyXSxbMiwzLCJcXERlbHRhIChmKSA6PSBcXHByb2Rfe2QgXFxpbiBcXG9iaihcXG1je0R9KX0gZiJdXQ==
    \[\begin{tikzcd}
    	c && {\Delta (c) := \prod_{d \in \obj(\mc{D})} c} \\
    	{c'} && {\Delta (c') := \prod_{d \in \obj(\mc{D})} c'}
    	\arrow["f"', from=1-1, to=2-1]
    	\arrow["{\Delta (f) := \prod_{d \in \obj(\mc{D})} f}", from=1-3, to=2-3]
    \end{tikzcd}
    \text{.}
    \]
\end{remark}

\begin{remark}[Usage]
    The diagonal functor is often used in precomposition with other functors to obtain constant sequences (e.g. in tensor algebra). 
\end{remark}

\section{Comma category}

\begin{definition}[Comma category]
Given functors

$$\mc{A} \xto{S} \mc{C} \xfrom{T} \mc{B},$$

the category $(S \downarrow T)$ is defined as follows:

\begin{itemize}
\item[$\bullet$] Objects of are the triples $(a: \obj(\mc{A}), h:\arr(\mc{C}), b: \obj(\mc{B})),$

\item[$\bullet$] Morphisms are pairs $(f: \arr(\mc{A}), g:\arr(\mc{B}))$ satisfying the diagram:
% https://q.uiver.app/#q=WzAsNixbMCwwLCIoYSwgaCwgYikiXSxbMCwxLCIoYScsIGgnLCBiJykiXSxbMSwwLCJTYSJdLFsxLDEsIlNhJyJdLFsyLDAsIlRiIl0sWzIsMSwiVGInIl0sWzAsMSwiKGYsIGcpIl0sWzIsNCwiaCJdLFszLDUsImgnIiwyXSxbMiwzLCJTZiIsMl0sWzQsNSwiVGYiXV0=
\[\begin{tikzcd}
	{(a, h, b)} & Sa & Tb \\
	{(a', h', b')} & {Sa'} & {Tb'}
	\arrow["{(f, g)}", from=1-1, to=2-1]
	\arrow["h", from=1-2, to=1-3]
	\arrow["Sf"', from=1-2, to=2-2]
	\arrow["Tg", from=1-3, to=2-3]
	\arrow["{h'}"', from=2-2, to=2-3]
\end{tikzcd}
\text{,}
\]

\item[$\bullet$] The composition is defined componentwise: 
% https://q.uiver.app/#q=WzAsOSxbMCwwLCIoYSwgaCwgYikiXSxbMSwwLCIoYScsIGgnLCBiJykiXSxbMSwxLCIoYScnLCBoJycsIGInJykiXSxbMywwLCJTYSJdLFs1LDEsIlRiJyciXSxbMywxLCJUYiJdLFs0LDEsIlRiJyJdLFs1LDAsIlNhJyciXSxbNCwwLCJTYSciXSxbMCwxLCIoZiwgZykiXSxbMSwyLCIoZicsIGcnKSJdLFswLDIsIihnIFxcY2lyYyBmLCBnJyBcXGNpcmMgZicpIiwyLHsibGFiZWxfcG9zaXRpb24iOjQwfV0sWzMsOCwiU2YiXSxbOCw3LCJTZiciXSxbNSw2LCJUZyIsMl0sWzYsNCwiVGcnIiwyXSxbMyw1LCJoIiwyXSxbOCw2LCJoJyIsMl0sWzcsNCwiaCcnIiwyXV0=
\[\begin{tikzcd}
	{(a, h, b)} & {(a', h', b')} && Sa & {Sa'} & {Sa''} \\
	& {(a'', h'', b'')} && Tb & {Tb'} & {Tb''}
	\arrow["{(f, g)}", from=1-1, to=1-2]
	\arrow["{(g \circ f, g' \circ f')}"'{pos=0.4}, from=1-1, to=2-2]
	\arrow["{(f', g')}", from=1-2, to=2-2]
	\arrow["Sf", from=1-4, to=1-5]
	\arrow["h"', from=1-4, to=2-4]
	\arrow["{Sf'}", from=1-5, to=1-6]
	\arrow["{h'}"', from=1-5, to=2-5]
	\arrow["{h''}"', from=1-6, to=2-6]
	\arrow["Tg"', from=2-4, to=2-5]
	\arrow["{Tg'}"', from=2-5, to=2-6]
\end{tikzcd}
\text{.}
\]
\end{itemize}

\end{definition}

\begin{remark}
    \hfill
    \begin{itemize}
        \item $(f, g) = (f', g')$ iff $f = f'$ and $g = g'$ -- there's no quotienting by $S$ or $T$;
        \item Thus, $(a, h, b) \simeq (a', h', b')$ iff $f$ and $g$ are isos.
    \end{itemize}
\end{remark}

\begin{definition}[(Co)Slice category]
\hfill
    \begin{itemize}
        \item Slice category. Let $S = \id_\mc{A}$, $\mc{B} = 1$:
    $$\mc{A} \xto{\id_\mc{A}} \mc{A} \xfrom{b} 1.$$
    % https://q.uiver.app/#q=WzAsNSxbMCwwLCIoYSxoICkgOj0oYSwgaCwgKikiXSxbMCwxLCIoYScsIGgpIDo9KGEnLCBoJywgKikiXSxbMiwwLCJhIl0sWzMsMCwiYiJdLFsyLDEsImEnIl0sWzAsMSwiZiA6PSAoZiwgXFxpZF8qKSJdLFsyLDQsImYiLDJdLFs0LDMsImgnIiwyXSxbMiwzLCJoIl1d
    \[\begin{tikzcd}
    	{(a,h ) :=(a, h, *)} && a & b \\
    	{(a', h) :=(a', h', *)} && {a'}
    	\arrow["{f := (f, \id_*)}", from=1-1, to=2-1]
    	\arrow["h", from=1-3, to=1-4]
    	\arrow["f"', from=1-3, to=2-3]
    	\arrow["{h'}"', from=2-3, to=1-4]
    \end{tikzcd}  
    \text{;}
    \]
    \item Coslice category. Let $T = \id_\mc{B}$, $\mc{A} = 1$:
    $$1 \xto{a} \mc{B} \xfrom{\id_\mc{B}} \mc{B},$$
    % https://q.uiver.app/#q=WzAsNSxbMCwwLCIoYixoICkgOj0oKiwgaCxiKSJdLFswLDEsIihiJywgaCkgOj0oKiwgaCcsIGInKSJdLFszLDAsImIiXSxbMywxLCJiJyJdLFsyLDAsImEiXSxbMCwxLCJnIDo9IChnLCBcXGlkXyopIl0sWzQsMiwiaCJdLFs0LDMsImgnIiwyXSxbMiwzLCJnIl1d
    \[\begin{tikzcd}
    	{(b,h ) :=(*, h,b)} && a & b \\
    	{(b', h) :=(*, h', b')} &&& {b'}
    	\arrow["{g := (g, \id_*)}", from=1-1, to=2-1]
    	\arrow["h", from=1-3, to=1-4]
    	\arrow["{h'}"', from=1-3, to=2-4]
    	\arrow["g", from=1-4, to=2-4]
    \end{tikzcd}
    \text{.}
    \]
        \end{itemize}
\end{definition}

\begin{definition}[Arrow category]
    Let $S = T = \id_\mc{C}$:
    $$\mc{C} \xto{\id_\mc{C}} \mc{C} \xfrom{\id_\mc{C}} \mc{C},$$
% https://q.uiver.app/#q=WzAsNixbMCwwLCIoYSwgaCwgYikiXSxbMCwxLCIoYScsIGgnLCBiJykiXSxbMiwwLCJhIl0sWzIsMSwiYSciXSxbMywwLCJiIl0sWzMsMSwiYiciXSxbMCwxLCIoZiwgZykiXSxbMiwzLCJmIiwyXSxbNCw1LCJnIl0sWzIsNCwiaCJdLFszLDUsImgnIiwyXV0=
\[\begin{tikzcd}
	{(a, h, b)} && a & b \\
	{(a', h', b')} && {a'} & {b'}
	\arrow["{(f, g)}", from=1-1, to=2-1]
	\arrow["h", from=1-3, to=1-4]
	\arrow["f"', from=1-3, to=2-3]
	\arrow["g", from=1-4, to=2-4]
	\arrow["{h'}"', from=2-3, to=2-4]
\end{tikzcd}
\text{.}
\]
    
\end{definition}

\end{document}
