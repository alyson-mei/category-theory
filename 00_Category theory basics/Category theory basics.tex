\documentclass[12pt]{article}
% Page setup
\usepackage[margin=1in]{geometry}
\usepackage{parskip}
\usepackage{titling}
\setlength{\droptitle}{-6em} 

% Math packages
\usepackage{amsmath, amsthm, amssymb, amsfonts}
\usepackage{mathtools}
\makeatletter
\g@addto@macro\th@plain{\normalfont} % remove italics from theorems (plain style)
\makeatother

% Commutative diagrams and proofs
\usepackage{tikz-cd}
\usepackage{proof}


% Theorem environments
\newtheorem{theorem}{Theorem}[section]
\newtheorem{lemma}[theorem]{Lemma}
\newtheorem{proposition}[theorem]{Proposition}
\newtheorem{corollary}[theorem]{Corollary}
\newtheorem{fact}[theorem]{Fact}
\newtheorem{remark}[theorem]{Remark}

\theoremstyle{definition}
\newtheorem{definition}[theorem]{Definition}
\newtheorem{example}[theorem]{Example}

% Useful commands for category theory
\newcommand{\mc}{\mathcal}
\newcommand{\ini}{\operatorname{ini}}
\newcommand{\ter}{\operatorname{ter}}
\newcommand{\cod}{\operatorname{cod}}
\newcommand{\dom}{\operatorname{dom}}

\newcommand{\List}{\operatorname{List}}
\newcommand{\lop}{\operatorname{[ \ ]}}
\newcommand{\len}{\operatorname{len}}
\newcommand{\type}{\operatorname{type}}

\newcommand{\Exc}{\operatorname{Exception}}

\newcommand{\Kl}{\operatorname{Kl}}
\newcommand{\plus}{\operatorname{+}}

\newcommand{\Cat}{\mathcal{C}at}
\newcommand{\Vect}{\mathrm{Vect}_\mathbf{k}}
\newcommand{\Hom}{\operatorname{Hom}}
\newcommand{\Nat}{\operatorname{Nat}}
\newcommand{\id}{\mathrm{id}}
\newcommand{\op}{\mathrm{op}}
\newcommand{\obj}{\operatorname{obj}}
\newcommand{\arr}{\operatorname{arr}}
\newcommand{\adj}{\operatorname{adj}}
\newcommand{\coadj}{\operatorname{coadj}}
\newcommand{\colim}{\operatorname*{colim}}

% Arrows
\newcommand{\ra}{\rightarrow}
\newcommand{\from}{\leftarrow}
\newcommand{\To}{\Rightarrow}
\newcommand{\xto}{\xrightarrow}
\newcommand{\xfrom}{\xleftarrow}

\renewcommand{\labelitemi}{--} % Level 1: Dashes
\renewcommand{\labelitemii}{$\circ$} % Level 2: Circles
\renewcommand{\labelitemiii}{$\bullet$} % Level 3: Bullets (default)


\title{Category theory basics}
\author{Alyson Mei}
\date{\today}


\begin{document}

\maketitle

\tableofcontents

\newpage

\section{Categories}

\subsection{Basic definitions}

Category theory extensively uses the language of commutative diagrams.

\begin{definition}[Diagram]
A diagram consists of a directed graph whose vertices are objects and whose edges are maps
between these objects.
\end{definition}

\begin{definition}[Commutative diagram]
   A diagram is called commutative if for any two directed paths with the same starting object
   and the same ending object, the corresponding compositions of maps are equal.

\end{definition}
\paragraph{Conventions:}
\begin{itemize}
    \item All diagrams shown are assumed to be commutative by default;
    \item By ``satisfying the diagrams'' we mean ``such that the diagrams commute''.
\end{itemize}
\begin{example} Commutativity of the diagram below means that $g \circ f = h \circ p$:
    % https://q.uiver.app/#q=WzAsNCxbMCwwLCJBIl0sWzEsMCwiQiJdLFsxLDEsIkQiXSxbMCwxLCJDIl0sWzAsMSwiZiJdLFswLDMsInAiLDJdLFsxLDIsImciXSxbMywyLCJoIiwyXV0=
    \[\begin{tikzcd}
    	A & B \\
    	C & D
    	\arrow["f", from=1-1, to=1-2]
    	\arrow["p"', from=1-1, to=2-1]
    	\arrow["g", from=1-2, to=2-2]
    	\arrow["h"', from=2-1, to=2-2]
    \end{tikzcd}
    \text{.}
    \]
\end{example}


\begin{definition}[Category]
    A category $\mc{C}$ consists of:
    \begin{itemize}
        \item a collection $\obj(\mc{C})$ of objects,
        \item a collection $\arr(\mc{C})$ of morphisms (arrows) of the form $f: C \to D$, where $C, D: \obj(\mc{C})$, 
    \end{itemize}
    satisfying the following diagrams for any morphisms $f, g, h$ of compatible types:
    \begin{itemize}
        \item[$\bullet$] Composition: 
             % https://q.uiver.app/#q=WzAsMyxbMCwwLCJBIl0sWzEsMCwiQiJdLFsxLDEsIkMiXSxbMCwxLCJmIl0sWzEsMiwiZyJdLFswLDIsImcgXFxjaXJjIGYiLDJdXQ==
            \[\begin{tikzcd}
            	A & B \\
            	& C
            	\arrow["f", from=1-1, to=1-2]
            	\arrow["{g \circ f}"', from=1-1, to=2-2]
            	\arrow["g", from=1-2, to=2-2]
            \end{tikzcd}
            \text{,}
        \]
        \item[$\bullet$] Identity:
            % https://q.uiver.app/#q=WzAsNixbMCwwLCJBIl0sWzEsMCwiQSJdLFsxLDEsIkIiXSxbMywxLCJBIl0sWzIsMCwiQyJdLFsyLDEsIkEiXSxbMSwyLCJmIl0sWzAsMiwiZiBcXGNpcmMgXFxpZF9BIiwyXSxbNSwzLCJcXGlkX0EiLDJdLFs0LDUsImciLDJdLFs0LDMsIlxcaWRfQSBcXGNpcmMgZyJdLFswLDEsIlxcaWRfQSJdXQ==
            \[\begin{tikzcd}
            	A & A & C \\
            	& B & A & A
            	\arrow["{\id_A}", from=1-1, to=1-2]
            	\arrow["{f \circ \id_A = f}"', from=1-1, to=2-2]
            	\arrow["f", from=1-2, to=2-2]
            	\arrow["g"', from=1-3, to=2-3]
            	\arrow["{\id_A \circ g = g}", from=1-3, to=2-4]
            	\arrow["{\id_A}"', from=2-3, to=2-4]
            \end{tikzcd}
            \text{,}
            \]
        \item[$\bullet$] Associativity:
            % https://q.uiver.app/#q=WzAsNCxbMCwwLCJBIl0sWzEsMCwiQiJdLFsxLDEsIkMiXSxbMiwxLCJEIl0sWzAsMSwiZiJdLFsxLDIsImciXSxbMiwzLCJoIiwyXSxbMCwyLCJnIFxcY2lyYyBmIiwyXSxbMSwzLCJoIFxcY2lyYyBnIl1d
            \[\begin{tikzcd}
            	A & B \\
            	& C & D
            	\arrow["f", from=1-1, to=1-2]
            	\arrow["{g \circ f}"', from=1-1, to=2-2]
            	\arrow["g", from=1-2, to=2-2]
            	\arrow["{h \circ g}", from=1-2, to=2-3]
            	\arrow["h"', from=2-2, to=2-3]
            \end{tikzcd}
            \text{.}
            \]
            \end{itemize}
\end{definition}

\begin{remark}
    \hfill
    \begin{itemize}
        \item The class of morphisms between objects $A$ and $B$ is often denoted as 
        $$\mc{C}(A,B) \text{ or } \Hom_\mc{C}(A, B);$$
        \item In some contexts, it is convenient to represent objects by their identity morphisms. 
        In such perspective, a category can be described in terms of arrows $f: A \to B$, where the objects are implicitly identified with their identities.
    \end{itemize}
\end{remark}


\subsection{Additional definitions}

\begin{definition}[(Co)Domain]
    Let $f: A \to B$ be a morphism in a category $\mc{C}$. 
    The object $A$ from which $f$ originates is called the domain of $f$, 
    and the object $B$ at which $f$ terminates is called the codomain of $f$.


Notation:
$$\dom(f), \ \cod(f).$$
\end{definition}

\begin{definition}[Mono, epi, iso]
    Consider a category $\mc{C}$.
    \begin{itemize}
        \item A morphism $m: \arr(\mc{C})$ is called a monomorphism (mono) if it is left-cancellative:% https://q.uiver.app/#q=WzAsNCxbMCwwLCJBIl0sWzEsMCwiQiJdLFsyLDAsIkMsIl0sWzQsMCwibVxcY2lyYyAgZl8xID0gbSBcXGNpcmMgZl8yICBcXFJpZ2h0YXJyb3cgZl8xID0gZl8yIl0sWzAsMSwiZl8yIiwyLHsib2Zmc2V0IjoyfV0sWzAsMSwiZl8xIiwwLHsib2Zmc2V0IjotMn1dLFsxLDIsIm0iXV0=
        \[\begin{tikzcd}
        	A & B & {C,} && {m\circ  f_1 = m \circ f_2  \Rightarrow f_1 = f_2}
        	\arrow["{f_2}"', shift right=2, from=1-1, to=1-2]
        	\arrow["{f_1}", shift left=2, from=1-1, to=1-2]
        	\arrow["m", from=1-2, to=1-3]
        \end{tikzcd}
        \text{;}\]
        \item A morphism $e: \arr(\mc{C})$ is called an epimorphism (epi) if it is right-cancellative:
        % https://q.uiver.app/#q=WzAsNCxbNCwwLCJnXzEgXFxjaXJjIGUgPSBnXzIgXFxjaXJjIGUgXFxSaWdodGFycm93IGdfMSA9IGdfMiJdLFswLDAsIkEiXSxbMSwwLCJCIl0sWzIsMCwiQywiXSxbMSwyLCJlIl0sWzIsMywiZ18xIiwwLHsib2Zmc2V0IjotMn1dLFsyLDMsImdfMiIsMix7Im9mZnNldCI6Mn1dXQ==
        \[\begin{tikzcd}
        	A & B & {C,} && {g_1 \circ e = g_2 \circ e \Rightarrow g_1 = g_2}
        	\arrow["e", from=1-1, to=1-2]
        	\arrow["{g_1}", shift left=2, from=1-2, to=1-3]
        	\arrow["{g_2}"', shift right=2, from=1-2, to=1-3]
        \end{tikzcd}\text{;}\]
        \item A morphism $i: \arr(\mc{C})$ is called  an isomorphism (iso) if has an inverse $i^{-1}: \arr(\mc{C})$:
        % https://q.uiver.app/#q=WzAsNixbMSwwLCJCIl0sWzAsMCwiQSJdLFsxLDEsIkEiXSxbMywwLCJCIl0sWzMsMSwiQSJdLFs0LDEsIkIiXSxbMSwwLCJpIl0sWzAsMiwiaV57LTF9Il0sWzEsMiwiXFxpZF9BIiwyXSxbMyw1LCJcXGlkX0IiXSxbMyw0LCJpXnstMX0iLDJdLFs0LDUsImkiLDJdXQ==
        \[\begin{tikzcd}
        	A & B && B \\
        	& A && A & B
        	\arrow["i", from=1-1, to=1-2]
        	\arrow["{\id_A}"', from=1-1, to=2-2]
        	\arrow["{i^{-1}}", from=1-2, to=2-2]
        	\arrow["{i^{-1}}"', from=1-4, to=2-4]
        	\arrow["{\id_B}", from=1-4, to=2-5]
        	\arrow["i"', from=2-4, to=2-5]
        \end{tikzcd}\text{.}\]
    \end{itemize}
\end{definition}

\begin{remark}
\hfill
    \begin{itemize}
        \item Mono, epi and iso are the generalizations of injection, surjection and bijection respectively;
        \item Every isomorphism is both mono and epi, but the converse is not generally true.
    \end{itemize}
\end{remark}

\begin{definition}[Initial, terminal and zero objects] Consider a category $\mc{C}$. 

    \begin{itemize}
        \item $I: \obj(\mc{C})$ is called initial object if for every $C: \obj(\mc{C})$ there exists a unique morphism $I \to C$;
        \item $T: \obj(\mc{C})$ is called terminal object if for every $C: \obj(\mc{C})$ there exists a unique morphism $C \to T$;
        \item $0: \obj({\mc{C}})$ is called zero object if it is both initial and terminal.
    \end{itemize}
    An object is called universal if it is initial or terminal.
\end{definition}

\begin{remark}
\hfill  
    \begin{itemize}
        \item Every two initial objects / every two terminal objects are isomorphic to each other;
        \item The concept of universal object combined with comma categories allows to gracefully define the variety of categorical structures.
    \end{itemize}
\end{remark}

\begin{definition}[Small and locally small categories]
    The category is called:
    \begin{itemize}
        \item Small, if both objects and morphisms form a set;
        \item Locally small, if morphisms between any two objects form a set.
    \end{itemize}
\end{definition}

\begin{definition}[Subcategory, full subcategory]
    Consider the category $\mc{C}$. A category $\mc{D}$ is called a subcategory of $\mc{C}$ if
    $\arr(\mc{D}) \subseteq \arr(\mc{C}).$
    It is called a full subcategory if, for any $A,B \in \obj(\mc{D})$, every morphism $A \to B$ in $\mc{C}$ is also in $\arr(\mc{D})$.
\end{definition}

\subsection{List of examples}

This list presents examples of categories from different branches of mathematics and introduces notation for some frequently used categories.

\begin{example}
\hfill
\begin{itemize}
    \item Trivial and finite categories:
    \begin{itemize}
        \item The empty category $0$;
        \item The terminal category $1$:
        $\obj(1) = \{*\}$, $\arr(1) = \{\id_*\}$;
        \item A monoid: a category with a single object;
        \item The category $2$ with two objects and a single non-identity arrow;
        \item The discrete category $\mc{C}_X$ on a set $X$:
        $\obj(\mc{C}_X) = X$, $\arr(\mc{C}_X) = \{\id_x \mid x \in X\}$;
        \item The diagram category $\mc{C}_G$ associated to a directed graph $G$:
        $\obj(\mc{C}_G) = \mathrm{vertices}(G)$, $\arr(\mc{C}_G) = \mathrm{edges}(G)$,
        with all compatible paths commuting;
    \end{itemize}
    \item Order-theoretic categories:
    \begin{itemize}
        \item Preorders as categories: objects are elements of preorder $P$, and there is a unique arrow $x \to y$ whenever $x \leq y$;
        \item Preorders / Lattices / Boolean algebras: same as above, only the arrows satisfy additional axioms;
    \end{itemize}
    \item Basic concrete categories:
    \begin{itemize}
        \item $\mathcal{S}et$: objects are sets, morphisms are functions;
        \item $\mathcal{F}in\mathcal{S}et$: objects are finite sets, morphisms are functions;
        \item $\mathcal{R}el$: objects are sets, morphisms are binary relations;
        \item $\mathcal{P}art$: objects are sets, morphisms are partial functions;
        \item $\mathcal{P}oint$: objects are pointed sets (sets with a distinguished element), morphisms are functions preserving the base point;
    \end{itemize}
    \item Logical / type-theoretic categories:
    \begin{itemize}
        \item Boolean algebras / Heyting algebras: objects are Boolean or Heyting algebras, morphisms are structure-preserving functions;
        \item Cartesian closed categories: categories with finite products and exponentials;
        \item Syntactic categories: constructed from logical theories or type theories; objects are formulas or types, morphisms are proofs or terms;
        \item Toposes: categories with certain logical and categorical structure supporting an internal logic;
    \end{itemize}
    \item Algebraic categories:
    \begin{itemize}
        \item $\mc{M}on$: monoids and monoid homomorphisms;
        \item $\mc{G}rp$: groups and group homomorphisms;
        \item $\mc{A}b$: abelian groups;
        \item $(\mc{C})\mc{R}ing$: (commutative) rings and ring homomorphisms;
        \item $\mc{V}ect_\mathbf{k}$: vector spaces over a field 
k and linear maps;
        \item $\mc{A}lg_\mathbf{k}$: algebras over a field or ring $\mathbf{k}$;
        \item $\mc{M}od_R$: modules over a ring $R$;
        \item $\mc{L}ie\mc{A}lg$ and $\mc{H}opf\mc{A}lg$: Lie and Hopf algebras and structure preserving maps;
    \end{itemize}
    \item Topological / geometric categories:
    \begin{itemize}
        \item $\mc{T}op$: topological spaces with continuous maps;
        \item $\mc{T}op_*$: pointed spaces with basepoint-preserving maps;
        \item $h\mc{T}op$: homotopy category; morphisms are homotopy classes of continuous maps;
        \item $\mc{M}an$: smooth manifolds with smooth maps;
        \item $\mc{D}iff$: diffeological spaces with smooth maps;
        \item $\mc{S}ch$: schemes and morphisms of schemes;
        \item $\mc{V}ar_\mathbf{k}$: algebraic varieties over a field $\mathbf{k}$;
        \item $\mc{C}ob$: cobordisms; objects are manifolds, morphisms are cobordism classes;
    \end{itemize}
    \item Higher level constructions:
        \begin{itemize}
            \item $\mc{C}at$: small categories and functors between them;
            \item $\mc{CAT}$: (large) categories and functors — “set of all categories” is only valid if we allow a universe distinction to avoid size issues;
            \item $\mc{G}rpd$: groupoids — categories in which all morphisms are invertible;
            \item $\mc{M}on\mc{C}at$: monoidal categories and monoidal functors;
            \item $\mc{A}b\mc{C}at$: abelian categories and additive functors;
            \item $[\mc{C}, \mc{D}]$: functor category; objects are functors $\mc{C} \to \mc{D}$, morphisms are natural transformations;
            \item Comma categories $(F \downarrow G)$: objects are triples $(C, D, f : F(C) \to G(D))$, morphisms are pairs making the corresponding squares commute.
        \end{itemize}
    \end{itemize}
    
\end{example}

\section{Functors}

\begin{definition}[Functor]
    Consider categories $\mc{C}$ and $\mc{D}$. 
    A functor $F : \mc{C} \to \mc{D}$ is a map from $\mc{C}$ to $\mc{D}$ that preserves identities and composition:

% https://q.uiver.app/#q=WzAsNCxbMCwwLCJBIl0sWzAsMSwiQSJdLFsyLDAsIkZBIl0sWzIsMSwiRkEiXSxbMCwxLCJcXGlkX0EiLDJdLFsyLDMsIkZcXGlkX0EgPSBcXGlkX3tGQX0iLDJdXQ==
\[\begin{tikzcd}
	A && FA \\
	A && FA
	\arrow["{\id_A}"', from=1-1, to=2-1]
	\arrow["{F\id_A = \id_{FA}}"', from=1-3, to=2-3]
\end{tikzcd}
\text{;}\]
    % https://q.uiver.app/#q=WzAsNixbMywwLCJGQSJdLFs0LDAsIkZCIl0sWzAsMCwiQSJdLFsxLDAsIkIiXSxbMSwxLCJDIl0sWzQsMSwiRkMiXSxbMCwxLCJGZiJdLFsyLDMsImYiXSxbMyw0LCJnIl0sWzIsNCwiZyBcXGNpcmMgZiIsMl0sWzAsNSwiRihnXFxjaXJjIGYpID0gRmcgXFxjaXJjIEZmIiwyXSxbMSw1LCJGZyJdXQ==
\[\begin{tikzcd}
	A & B && FA & FB \\
	& C &&& FC
	\arrow["f", from=1-1, to=1-2]
	\arrow["{g \circ f}"', from=1-1, to=2-2]
	\arrow["g", from=1-2, to=2-2]
	\arrow["Ff", from=1-4, to=1-5]
	\arrow["{F(g\circ f) = Fg \circ Ff}"', from=1-4, to=2-5]
	\arrow["Fg", from=1-5, to=2-5]
\end{tikzcd}
\text{.}
\]
\end{definition}

\begin{remark}
    The image of a functor is not necessarily a category.
\end{remark}

\begin{definition}[Full, faithful and fully faithful functors]
The functor is called:
\begin{itemize}
    \item Faithful, if it is injective on morphisms;
    \item Full, if it is surjective on morphisms;
    \item Fully faithful, if it is both full and faithful.
\end{itemize}
\end{definition}

\section{Natural transformations}

\begin{definition}[Natural transformation] Let $F, G: \mc{C} \to \mc{D}$. A natural transformation $\alpha$ between $F$ and $G$ is a collection of morphisms $(\alpha_c \mid c \in \obj(\mathcal{C}))$ satisfying the diagrams
% https://q.uiver.app/#q=WzAsNixbMCwwLCJjIl0sWzAsMSwiYyciXSxbMiwwLCJGYyJdLFsyLDEsIkZjJyJdLFszLDAsIkdjIl0sWzMsMSwiR2MnIl0sWzAsMSwiZiIsMl0sWzIsMywiRmYiLDJdLFs0LDUsIkdmIl0sWzIsNCwiXFxhbHBoYV9jIl0sWzMsNSwiXFxhbHBoYV97Yyd9IiwyXV0=
\[\begin{tikzcd}
	c && Fc & Gc \\
	{c'} && {Fc'} & {Gc'}
	\arrow["f"', from=1-1, to=2-1]
	\arrow["{\alpha_c}", from=1-3, to=1-4]
	\arrow["Ff"', from=1-3, to=2-3]
	\arrow["Gf", from=1-4, to=2-4]
	\arrow["{\alpha_{c'}}"', from=2-3, to=2-4]
\end{tikzcd}
\text{.}
\]
Notation:
% https://q.uiver.app/#q=WzAsNCxbMCwwLCJGIl0sWzEsMCwiRyJdLFszLDAsIlxcbWF0aGNhbHtDfSJdLFs0LDAsIlxcbWF0aGNhbHtEfSJdLFswLDEsIlxcYWxwaGEiXSxbMiwzLCJGIiwwLHsiY3VydmUiOi0yfV0sWzIsMywiRyIsMix7ImN1cnZlIjoyfV0sWzUsNiwiXFxhbHBoYSIsMCx7InNob3J0ZW4iOnsic291cmNlIjoyMCwidGFyZ2V0IjoyMH19XV0=
\[\begin{tikzcd}
	F & G && {\mathcal{C}} & {\mathcal{D}}
	\arrow["\alpha", from=1-1, to=1-2]
	\arrow[""{name=0, anchor=center, inner sep=0}, "F", bend left = 36pt, from=1-4, to=1-5]
	\arrow[""{name=1, anchor=center, inner sep=0}, "G"', bend right = 36pt, from=1-4, to=1-5]
	\arrow["\alpha", shorten <=3pt, shorten >=3pt, Rightarrow, from=0, to=1]
\end{tikzcd}
\text{.}
\]

\end{definition}



\end{document}
