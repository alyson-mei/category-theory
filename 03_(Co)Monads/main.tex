\documentclass[12pt]{article}
% Page setup
\usepackage[margin=1in]{geometry}
\usepackage{parskip}
\usepackage{titling}
\setlength{\droptitle}{-6em} 

% Math packages
\usepackage{amsmath, amsthm, amssymb, amsfonts}
\usepackage{mathtools}
\makeatletter
\g@addto@macro\th@plain{\normalfont} % remove italics from theorems (plain style)
\makeatother

% Commutative diagrams
\usepackage{tikz-cd}


% Theorem environments
\newtheorem{theorem}{Theorem}[section]
\newtheorem{lemma}[theorem]{Lemma}
\newtheorem{proposition}[theorem]{Proposition}
\newtheorem{corollary}[theorem]{Corollary}
\newtheorem{fact}[theorem]{Fact}
\newtheorem{remark}[theorem]{Remark}

\theoremstyle{definition}
\newtheorem{definition}[theorem]{Definition}
\newtheorem{example}[theorem]{Example}

% Useful commands for category theory
\newcommand{\mc}{\mathcal}
\newcommand{\ini}{\operatorname{ini}}
\newcommand{\ter}{\operatorname{ter}}

\newcommand{\List}{\operatorname{List}}
\newcommand{\lop}{\operatorname{[ \ ]}}
\newcommand{\len}{\operatorname{len}}

\newcommand{\Exc}{\operatorname{Exception}}

\newcommand{\Kl}{\operatorname{Kl}}
\newcommand{\plus}{\operatorname{+}}

\newcommand{\Cat}{\mathcal{C}at}
\newcommand{\Vect}{\mathrm{Vect}_\mathbf{k}}
\newcommand{\Hom}{\operatorname{Hom}}
\newcommand{\Nat}{\operatorname{Nat}}
\newcommand{\id}{\mathrm{id}}
\newcommand{\op}{\mathrm{op}}
\newcommand{\obj}{\operatorname{obj}}
\newcommand{\arr}{\operatorname{arr}}
\newcommand{\adj}{\operatorname{adj}}
\newcommand{\coadj}{\operatorname{coadj}}
\newcommand{\colim}{\operatorname*{colim}}

% Arrows
\newcommand{\ra}{\rightarrow}
\newcommand{\from}{\leftarrow}
\newcommand{\To}{\Rightarrow}
\newcommand{\xto}{\xrightarrow}
\newcommand{\xfrom}{\xleftarrow}

\renewcommand{\labelitemi}{--} % Level 1: Dashes
\renewcommand{\labelitemii}{$\circ$} % Level 2: Circles
\renewcommand{\labelitemiii}{$\bullet$} % Level 3: Bullets (default)


\title{(Co)Monads}
\author{Alyson Mei}
\date{\today}

\begin{document}

\maketitle

\tableofcontents


\paragraph{Conventions:}
\begin{itemize}
    \item Standard polymorphic notation is used freely when no ambiguity arises;
    \item All diagrams are assumed to commute unless stated otherwise.
\end{itemize}

% \paragraph{TODO:}
% \begin{itemize}
%     \item Specify the $\Delta$ framework;
%     \item Check the details for the Exception monad;
%     \item Consider algebras, Eilenberg–Moore category, etc.;
%     \item Write the comonads section.
% \end{itemize}


\newpage

\section{Monads}

\subsection{Monoid object and Monad}

\begin{definition}[Monoid object]

Let $(\mc{C}, \otimes, I)$ be a monoidal category. A monoid object in $\mc{C}$ consists of:
\begin{itemize}
    \item[$\bullet$] an object $M \in \obj(\mc{C})$,
    \item[$\bullet$] a multiplication morphism
    \[
        \mu : M \otimes M \to M,
    \]
    \item[$\bullet$] a unit morphism
    \[
        \eta : I \to M,
    \]
\end{itemize}
satisfying the unit and associativity diagrams:
% https://q.uiver.app/#q=WzAsOCxbMSwwLCJNIFxcdGltZXMgTSJdLFswLDAsIk0iXSxbMSwxLCJNIl0sWzIsMCwiTSJdLFs0LDAsIk1eMyJdLFs0LDEsIk1eMiJdLFs1LDAsIk1eMiJdLFs1LDEsIk0iXSxbMSwwLCJNIFxcdGltZXMgXFxldGEiXSxbMCwyLCJcXG11IiwxXSxbMSwyLCJcXGlkIiwyXSxbMywyLCJcXGlkIl0sWzMsMCwiXFxldGEgXFx0aW1lcyBNIiwyXSxbNCw1LCIgTSBcXHRpbWVzIFxcbXUiLDJdLFs0LDYsIlxcbXUgXFx0aW1lcyBNIl0sWzYsNywiXFxtdSJdLFs1LDcsIlxcbXUiLDJdXQ==
\[\begin{tikzcd}
	M & {M \otimes M} & M && {M^{\otimes 3}} & {M^{\otimes2}} \\
	& M &&& {M^{\otimes 2}} & M
	\arrow["{M \otimes \eta}", from=1-1, to=1-2]
	\arrow["\id"', from=1-1, to=2-2]
	\arrow["\mu"{description}, from=1-2, to=2-2]
	\arrow["{\eta \otimes M}"', from=1-3, to=1-2]
	\arrow["\id", from=1-3, to=2-2]
	\arrow["{\mu \otimes M}", from=1-5, to=1-6]
	\arrow["{ M \otimes \mu}"', from=1-5, to=2-5]
	\arrow["\mu", from=1-6, to=2-6]
	\arrow["\mu"', from=2-5, to=2-6]
\end{tikzcd}
\text{.}
\]
\end{definition}

\begin{example}
\hfill
\begin{itemize}
    \item Categories with finite products $(\mc{C}, \times, 1)$ are monoidal:
    \begin{itemize}
        \item In $\mc{S}et$, monoid objects are ordinary monoids;
        \item In $\Vect$, monoid objects are associative unital algebras over a field $\mathbf{k}$;
    \end{itemize}
    \item The category of endofunctors $([\mc{C}, \mc{C}], \circ, \id)$ is monoidal; monoid objects in this category are called monads.
\end{itemize}
\end{example}



\begin{definition}[Monad]
Monad is a triple $$(M: \mc{C} \to \mc{C}, \eta: \id_\mc{C} \to M, \mu: M^2 \to M),$$
satisfying the unit and associativity diagrams:
% https://q.uiver.app/#q=WzAsOCxbMSwwLCJNXjIiXSxbMiwwLCJNIl0sWzEsMSwiTSJdLFswLDAsIk0iXSxbMywwLCJNXjMiXSxbMywxLCJNXjIiXSxbNCwwLCJNXjIiXSxbNCwxLCJNIl0sWzMsMiwiXFxpZF9NIiwyXSxbMSwyLCJcXGlkX00iXSxbMCwyLCJcXG11Il0sWzMsMCwiXFxldGEgXFxjaXJjIE0iXSxbMSwwLCJNIFxcY2lyYyBcXGV0YSIsMl0sWzYsNywiXFxtdSJdLFs0LDUsIlxcbXUgXFxjaXJjIE0iLDJdLFs1LDcsIlxcbXUiLDJdLFs0LDYsIk0gXFxjaXJjIFxcbXUiXV0=
\[\begin{tikzcd}
	M & {M^2} & M & {M^3} & {M^2} \\
	& M && {M^2} & M
	\arrow["{\eta \circ M}", from=1-1, to=1-2]
	\arrow["{\id_M}"', from=1-1, to=2-2]
	\arrow["\mu", from=1-2, to=2-2]
	\arrow["{M \circ \eta}"', from=1-3, to=1-2]
	\arrow["{\id_M}", from=1-3, to=2-2]
	\arrow["{M \circ \mu}", from=1-4, to=1-5]
	\arrow["{\mu \circ M}"', from=1-4, to=2-4]
	\arrow["\mu", from=1-5, to=2-5]
	\arrow["\mu"', from=2-4, to=2-5]
\end{tikzcd}
\text{.}
\]
\end{definition}

\newpage

\subsection{Kleisli category}

\begin{definition}[Kleisli category]
Let $(T, \eta, \mu)$ be a monad on $\mc{C}$. The Kleisli category $\Kl(T)$  of the monad $T$ is defined as follows:
\begin{itemize}
    \item[$\bullet$] Objects of $\Kl(T)$ are the same as in $\mc{C},$
    \item [$\bullet$]
    For $A, B \in \obj(\mc{C})$, a morphism $f^\bullet: A \to B$ in $\Kl(T)$ is a morphism $f: A \to TB$ in $\mc{C}$,
    \item[$\bullet$] Composition:
% https://q.uiver.app/#q=WzAsMTAsWzAsMCwiQSJdLFsxLDAsIkIiXSxbMCwxLCJUQiJdLFsxLDEsIlRDIl0sWzMsMCwiQSJdLFsyLDAsIlRCIl0sWzMsMSwiVEMiXSxbMiwxLCJUXjJDIl0sWzUsMCwiQSJdLFs1LDEsIkMiXSxbMCwyLCJmIiwyXSxbMSwzLCJnIiwyXSxbNyw2LCJcXG11X0MiLDJdLFs0LDUsImYiLDJdLFs1LDcsIlRnIiwyXSxbNCw2LCJcXG11X0MgXFxjaXJjIFRnIFxcY2lyYyBmIl0sWzgsOSwiZl5cXGJ1bGxldCBcXGNpcmMgZ15cXGJ1bGxldCBcXGluIFxcS2woVCkiXV0=
\[\begin{tikzcd}
	A & B & TB & A && A \\
	TB & TC & {T^2C} & TC && C
	\arrow["f"', from=1-1, to=2-1]
	\arrow["g"', from=1-2, to=2-2]
	\arrow["Tg"', from=1-3, to=2-3]
	\arrow["f"', from=1-4, to=1-3]
	\arrow["{\mu_C \circ Tg \circ f}", from=1-4, to=2-4]
	\arrow["{f^\bullet \circ g^\bullet \in \Kl(T)}", from=1-6, to=2-6]
	\arrow["{\mu_C}"', from=2-3, to=2-4]
\end{tikzcd}
\text{,}\]
    \item[$\bullet$] Identity morphism $\id_{A}^\bullet$ in $\Kl(T)$ is a morphism $\eta_A: A \to TA$ in $\mc{C}:$
% https://q.uiver.app/#q=WzAsMTYsWzMsMCwiQSJdLFsyLDAsIlRBIl0sWzIsMSwiVF4yQiJdLFszLDEsIlRCIl0sWzAsMCwiQSJdLFswLDEsIlRCIl0sWzEsMCwiVEEiXSxbMSwxLCJUXjJCIl0sWzUsMCwiQSJdLFs1LDEsIkIiXSxbMywzLCJUQSJdLFsyLDMsIlReMkEiXSxbMywyLCJDIl0sWzIsMiwiVEEiXSxbNSwyLCJBIl0sWzUsMywiQiJdLFswLDEsIlxcZXRhX0EiLDJdLFsxLDIsIlRmIiwyXSxbMiwzLCJcXG11X0IiLDJdLFswLDMsIlxcbXVfQiBcXGNpcmMgXFxldGFfe1RCfSBcXGNpcmMgZiA9IGYiXSxbNSw3LCJcXGV0YV97VEJ9IiwyXSxbNCw1LCJmIiwyXSxbNCw2LCJcXGV0YV9BIl0sWzYsNywiVGYiXSxbOCw5LCJmIFxcY2lyYyBcXGlkXlxcYnVsbGV0X0EiXSxbMTEsMTAsIlxcbXVfQSIsMl0sWzEyLDEzLCJnIiwyXSxbMTMsMTEsIlRcXGV0YV9BIiwyXSxbMTIsMTAsIlxcbXVfQSBcXGNpcmMgVCBcXGV0YV9BIFxcY2lyYyBnID0gZyJdLFsxNCwxNSwiXFxpZF9BXlxcYnVsbGV0IFxcY2lyYyBnIl1d
\[\begin{tikzcd}
	A & TA & TA & A && A \\
	TB & {T^2B} & {T^2B} & TB && B \\
	&& TA & C && A \\
	&& {T^2A} & TA && B
	\arrow["{\eta_A}", from=1-1, to=1-2]
	\arrow["f"', from=1-1, to=2-1]
	\arrow["Tf", from=1-2, to=2-2]
	\arrow["Tf"', from=1-3, to=2-3]
	\arrow["{\eta_A}"', from=1-4, to=1-3]
	\arrow["{\mu_B \circ \eta_{TB} \circ f = f}", from=1-4, to=2-4]
	\arrow["{f \circ \id^\bullet_A}", from=1-6, to=2-6]
	\arrow["{\eta_{TB}}"', from=2-1, to=2-2]
	\arrow["{\mu_B}"', from=2-3, to=2-4]
	\arrow["{T\eta_A}"', from=3-3, to=4-3]
	\arrow["g"', from=3-4, to=3-3]
	\arrow["{\mu_A \circ T \eta_A \circ g = g}", from=3-4, to=4-4]
	\arrow["{\id_A^\bullet \circ g}", from=3-6, to=4-6]
	\arrow["{\mu_A}"', from=4-3, to=4-4]
\end{tikzcd} 
\text{.}
\]
\end{itemize}
\end{definition}

\newpage

\subsection{Monads and adjunctions (unfinished)}

\begin{theorem} [Monads $\approx$ adjunctions]
    Every adjunction gives rise to a monad, and every monad arises (up to equivalence) from an adjunction.
\end{theorem}

\paragraph{From adjunction to monad.}
\hfill \break
Let $F: \mc{C} \to \mc{D}$, $G: \mc{D} \to \mc{C}$, and $F$ be the left adjoint to $G$:
$$\eta: \id_\mc{C} \to GF, \ \varepsilon: FG \to \id_\mc{D},$$
% https://q.uiver.app/#q=WzAsNixbMCwwLCJGIl0sWzEsMCwiRkdGIl0sWzEsMSwiRiJdLFszLDAsIkciXSxbMiwwLCJHRkciXSxbMiwxLCJHIl0sWzAsMSwiRlxcZXRhICJdLFsxLDIsIlxcdmFyZXBzaWxvbiBGIl0sWzAsMiwiXFxpZCIsMl0sWzMsNSwiXFxpZCJdLFszLDQsIlxcZXRhIEciLDJdLFs0LDUsIkdcXHZhcmVwc2lsb24iLDJdXQ==
\[\begin{tikzcd}
	F & FGF & GFG & G \\
	& F & G
	\arrow["{F\eta }", from=1-1, to=1-2]
	\arrow["\id"', from=1-1, to=2-2]
	\arrow["{\varepsilon F}", from=1-2, to=2-2]
	\arrow["{G\varepsilon}"', from=1-3, to=2-3]
	\arrow["{\eta G}"', from=1-4, to=1-3]
	\arrow["\id", from=1-4, to=2-3]
\end{tikzcd}.\]

We define the underlying functor $M$ of the monad as
$$M := GF : \mc{C} \to \mc{C}.$$
Its unit is exactly $\eta$. We can define the multiplication $\mu: M^2 \to M$ using the counit:
$$\mu := G\varepsilon F.$$
The $\varepsilon$ here essentially collapses the forget-free structure from $GF$.

The unit and associativity diagrams:

% https://q.uiver.app/#q=WzAsNixbMCwwLCJHRiJdLFsxLDAsIkdGR0YiXSxbMSwxLCJHRiJdLFs0LDAsIkdGIl0sWzMsMCwiR0ZHRiJdLFszLDEsIkdGIl0sWzAsMSwiXFxldGEgR0YiXSxbMSwyLCJcXG11ID0gR1xcdmFyZXBzaWxvbiBGIl0sWzAsMiwiKEdcXHZhcmVwc2lsb24gXFxjaXJjIFxcZXRhIEcpIEYgPSBcXGlkIiwyLHsibGFiZWxfcG9zaXRpb24iOjMwLCJvZmZzZXQiOjJ9XSxbNCw1LCJcXG11ID0gR1xcdmFyZXBzaWxvbiBGIiwyXSxbMyw0LCJHRlxcZXRhIiwyXSxbMyw1LCJHIChcXHZhcmVwc2lsb24gRiBcXGNpcmMgRiBcXGV0YSApID0gXFxpZCIsMCx7ImxhYmVsX3Bvc2l0aW9uIjozMH1dXQ==
\[\begin{tikzcd}
	GF & GFGF && GFGF & GF \\
	& GF && GF
	\arrow["{\eta GF}", from=1-1, to=1-2]
	\arrow["{(G\varepsilon \circ \eta G) F = \id}"'{pos=0.3}, shift right=2, from=1-1, to=2-2]
	\arrow["{\mu = G\varepsilon F}", from=1-2, to=2-2]
	\arrow["{\mu = G\varepsilon F}"', from=1-4, to=2-4]
	\arrow["{GF\eta}"', from=1-5, to=1-4]
	\arrow["{G (\varepsilon F \circ F \eta ) = \id}"{pos=0.3}, from=1-5, to=2-4]
\end{tikzcd}
\text{,}
\]
% https://q.uiver.app/#q=WzAsNCxbMCwwLCJHRkdGR0YiXSxbMiwwLCJHRkdGIl0sWzAsMiwiR0ZHRiJdLFsyLDIsIkdGIl0sWzAsMiwiR0ZHIFxcdmFyZXBzaWxvbiBGIiwyXSxbMCwxLCJHIFxcdmFyZXBzaWxvbiBGR0YiXSxbMSwzLCJHIFxcdmFyZXBzaWxvbiBGIl0sWzIsMywiRyBcXHZhcmVwc2lsb24gRiIsMl1d
\[\begin{tikzcd}
	GFGFGF && GFGF \\
	\\
	GFGF && GF
	\arrow["{G \varepsilon FGF}", from=1-1, to=1-3]
	\arrow["{GFG \varepsilon F}"', from=1-1, to=3-1]
	\arrow["{G \varepsilon F}", from=1-3, to=3-3]
	\arrow["{G \varepsilon F}"', from=3-1, to=3-3]
\end{tikzcd}
\text{.}
\]
\newpage
\subsection{Examples}

\subsubsection{List monad}
    \paragraph{Definition and monad structure.}
    \hfill \break
    Consider the functor $\List : \mc{S}et \to \mc{S}et$, which sends a set to the set of all finite lists of its elements:
    % https://q.uiver.app/#q=WzAsNixbMiwwLCJcXExpc3QgIEEiXSxbMiwxLCJcXExpc3QgIEIiXSxbMCwwLCJBIl0sWzAsMSwiQiJdLFszLDAsIlthLCBiLCBjXSJdLFszLDEsIltmKGEpLCBmKGIpLCBmKGMpXSJdLFsyLDMsImYiXSxbMCwxLCJcXExpc3QgZiJdLFs0LDUsIlxcTGlzdCBmIiwwLHsic3R5bGUiOnsidGFpbCI6eyJuYW1lIjoibWFwcyB0byJ9fX1dXQ==
    \[\begin{tikzcd}
    	A && {\List  A} & {[a_1, a_2, ... a_n]} \\
    	B && {\List  B} & {[f(a_1), f(a_2), ..., f(a_n)]}
    	\arrow["f", from=1-1, to=2-1]
    	\arrow["{\List f}", from=1-3, to=2-3]
    	\arrow["{\List f}", maps to, from=1-4, to=2-4]
    \end{tikzcd}
    \text{.}
    \]
    
    From a categorical perspective, the functor $\List$ can be presented as follows:
    % https://q.uiver.app/#q=WzAsOCxbMCwwLCJcXG1je1N9ZXQiXSxbMSwwLCJcXG1je1N9ZXReXFxtYXRoYmJ7Tn0iXSxbMSwxLCJcXG1je1N9ZXReXFxtYXRoYmJ7Tn0iXSxbMCwxLCJcXG1je1N9ZXQiXSxbMywwLCJBIl0sWzQsMCwiQV5cXG1hdGhiYntOfSJdLFs0LDEsIihBXm4pX3tuIFxcaW4gXFxtYXRoYmJ7Tn19Il0sWzMsMSwiXFxjb3Byb2Rfe24gXFxpbiBcXG1hdGhiYntOfX0gQV5uIl0sWzAsMSwiXFxEZWx0YSJdLFswLDMsIlxcTGlzdCIsMl0sWzIsMywiXFxjb3Byb2Rfe24gXFxpbiBcXG1hdGhiYntOfX0gIl0sWzEsMiwiXFxwcm9kX3tuIFxcaW4gXFxtYXRoYmJ7Tn19XFx0aW1lc15uIl0sWzUsNiwiIiwwLHsic3R5bGUiOnsidGFpbCI6eyJuYW1lIjoibWFwcyB0byJ9fX1dLFs0LDUsIiIsMCx7InN0eWxlIjp7InRhaWwiOnsibmFtZSI6Im1hcHMgdG8ifX19XSxbNiw3LCIiLDAseyJzdHlsZSI6eyJ0YWlsIjp7Im5hbWUiOiJtYXBzIHRvIn19fV0sWzQsNywiXFxMaXN0IChBKSIsMix7InN0eWxlIjp7InRhaWwiOnsibmFtZSI6Im1hcHMgdG8ifX19XV0=
    \[\begin{tikzcd}
    	{\mc{S}et} & {\mc{S}et^\mathbb{N}} && A & {A^\mathbb{N}} \\
    	{\mc{S}et} & {\mc{S}et^\mathbb{N}} && {\coprod_{n \in \mathbb{N}} A^n} & {(A^n)_{n \in \mathbb{N}}}
    	\arrow["\Delta", from=1-1, to=1-2]
    	\arrow["\List"', from=1-1, to=2-1]
    	\arrow["{\prod_{n \in \mathbb{N}}\times^n}", from=1-2, to=2-2]
    	\arrow[maps to, from=1-4, to=1-5]
    	\arrow["{\List (A)}"', maps to, from=1-4, to=2-4]
    	\arrow[maps to, from=1-5, to=2-5]
    	\arrow["{\coprod_{n \in \mathbb{N}} }", from=2-2, to=2-1]
    	\arrow[maps to, from=2-5, to=2-4]
    \end{tikzcd}
    \text{.}\]
    
    
    This presentation gives rise to the API:
    
    \begin{itemize}
        \item The collection of constructors
        $$
        \lop_{n, A} :=  A^n \xto{i_n} \List A := \lambda a_1 \dots a_n. [a_1, \dots, a_n],
        $$
        in particular:
        \begin{itemize}
            \item The single-element list constructor
            $$
            \eta_A : A \to \List A := \lambda a. [a],
            $$
            \item The empty list constructor 
            $$
            \lop_0 : 1 \to \List A := \lambda a. [];
            $$
        \end{itemize}
        \item Length of a list via the power of $A$ in the coproduct:
        $$
        \len_A : \List A \to \mathbf{N};
        $$
        \item List concatenation via the natural isomorphism $A^n \times A^m \cong A^{n+m}$:
    % https://q.uiver.app/#q=WzAsOCxbMiwwLCJBXm0gXFx0aW1lcyBBXm4iXSxbMywwLCJcXExpc3QgQSBcXHRpbWVzIFxcTGlzdCBBIl0sWzIsMSwiQV57bSArIG59Il0sWzMsMSwiXFxMaXN0IEEiXSxbMCwwLCJcXGlkIFxcdGltZXMgXFxpZCJdLFsxLDAsIiBcXExpc3RcXHRpbWVzIFxcTGlzdCJdLFswLDEsIlxcaWQiXSxbMSwxLCJcXExpc3QiXSxbMCwxLCJpX20gXFx0aW1lcyBpX24iXSxbMCwyLCJcXGNvbmciLDJdLFsyLDMsImlfe20gKyBufSIsMl0sWzEsMywiK19BIl0sWzYsNywiaSIsMl0sWzUsNywiKyJdLFs0LDYsIlxcc2ltZXEiLDJdLFs0LDUsImkgXFx0aW1lcyBpIl1d
    \[\begin{tikzcd}
    	{\id \times \id} & { \List\times \List} & {A^m \times A^n} & {\List A \times \List A} \\
    	\id & \List & {A^{m + n}} & {\List A}
    	\arrow["{i \times i}", from=1-1, to=1-2]
    	\arrow["\simeq"', from=1-1, to=2-1]
    	\arrow["{+}", from=1-2, to=2-2]
    	\arrow["{i_m \times i_n}", from=1-3, to=1-4]
    	\arrow["\cong"', from=1-3, to=2-3]
    	\arrow["{+_A}", from=1-4, to=2-4]
    	\arrow["i"', from=2-1, to=2-2]
    	\arrow["{i_{m + n}}"', from=2-3, to=2-4]
    \end{tikzcd}
    \text{,}
    \]
    $$+_A: \List A \times \List A \to \List A := \lambda l m. [l_0, ... l_{\len(l) - 1},   m_0, ..., m_{\len(m)  - 1}],$$
    $$+_A : (\List A)^n \to \List A := \lambda l_0...l_n. l_0 + ... + l_n;$$
    \item List destructors: 
    $$\mu_A: \List(\List A) \to \List(A) := \lambda l. \sum_{i = 0}^{\len(l) - 1} l_i,$$
    $$\mu_A : \List^nA \to \List A := \mu_A^{\circ n}.$$
    \end{itemize}
    
    %Destructor is a left inverse to constructor. Using a polymorphic notation we can write:
    %$$  \plus \circ \lop =  \id.$$
    
    First of all, this API provides an internal monoid structure on lists: for a fixed set $A$, we have a monoid object
    $$
    (\List A, [], +_A)
    $$
    in $\mc{S}et$. The corresponding diagrams are straightforward in this case.
    
    Second, it provides a monad structure on $\List$:
    $$
    (\List, \eta: \id \to \List, \mu: \List^2 \to \List).
    $$
    
    The corresponding diagrams:
    % https://q.uiver.app/#q=WzAsNixbMCwwLCJcXExpc3QgQSJdLFsyLDAsIlxcTGlzdCAoXFxMaXN0IEEpIl0sWzIsMSwiXFxMaXN0IEEiXSxbMywwLCJsIl0sWzQsMCwiW2xdIl0sWzQsMSwibCA9IFtsXV8wIl0sWzAsMSwiXFxldGFfe1xcTGlzdCBBfSJdLFsxLDIsIlxcbXVfQSJdLFswLDIsIlxcaWQiLDJdLFs0LDUsIiIsMCx7InN0eWxlIjp7InRhaWwiOnsibmFtZSI6Im1hcHMgdG8ifX19XSxbMyw0LCIiLDAseyJzdHlsZSI6eyJ0YWlsIjp7Im5hbWUiOiJtYXBzIHRvIn19fV0sWzMsNSwiIiwyLHsic3R5bGUiOnsidGFpbCI6eyJuYW1lIjoibWFwcyB0byJ9fX1dXQ==
    \[\begin{tikzcd}
    	{\List A} && {\List (\List A)} & l & {[l]} \\
    	&& {\List A} && {l = [l]_0}
    	\arrow["{\eta_{\List A}}", from=1-1, to=1-3]
    	\arrow["\id"', from=1-1, to=2-3]
    	\arrow["{\mu_A}", from=1-3, to=2-3]
    	\arrow[maps to, from=1-4, to=1-5]
    	\arrow[maps to, from=1-4, to=2-5]
    	\arrow[maps to, from=1-5, to=2-5]
    \end{tikzcd}
    \text{,}\]
    % https://q.uiver.app/#q=WzAsNixbMCwwLCJcXExpc3QgKFxcTGlzdCBBKSJdLFsyLDAsIlxcTGlzdCBBIl0sWzAsMSwiXFxMaXN0IEEiXSxbNCwwLCJsIl0sWzMsMCwiW1tsXzBdLCAuLi4sIFtsX3tcXGxlbihsKSAtIDF9XV0iXSxbMywxLCJsID0gXFxzdW1fe2k9MH1ee1xcbGVuKGwpIC0gMX0gW2xfaV0iXSxbMSwyLCJcXGlkIl0sWzEsMCwiXFxMaXN0IFxcZXRhX0EiLDJdLFswLDIsIlxcbXVfQSIsMl0sWzQsNSwiIiwyLHsic3R5bGUiOnsidGFpbCI6eyJuYW1lIjoibWFwcyB0byJ9fX1dLFszLDQsIiIsMix7InN0eWxlIjp7InRhaWwiOnsibmFtZSI6Im1hcHMgdG8ifX19XSxbMyw1LCIiLDAseyJzdHlsZSI6eyJ0YWlsIjp7Im5hbWUiOiJtYXBzIHRvIn19fV1d
    \[\begin{tikzcd}
    	{\List (\List A)} && {\List A} & {[[l_0], ..., [l_{\len(l) - 1}]]} & l \\
    	{\List A} &&& {l = \sum_{i=0}^{\len(l) - 1} [l_i]}
    	\arrow["{\mu_A}"', from=1-1, to=2-1]
    	\arrow["{\List \eta_A}"', from=1-3, to=1-1]
    	\arrow["\id", from=1-3, to=2-1]
    	\arrow[maps to, from=1-4, to=2-4]
    	\arrow[maps to, from=1-5, to=1-4]
    	\arrow[maps to, from=1-5, to=2-4]
    \end{tikzcd}
    \text{,}\]
    % https://q.uiver.app/#q=WzAsOCxbMiwwLCJsIl0sWzMsMCwiWyBcXHN1bV97aiA9IDB9XntcXGxlbihsXzApIC0gMX0gKGxfMClfaiwgLi4uLCAgXFxzdW1fe2o9MH1ee1xcbGVuKGxfe1xcbGVuKGwpIC0gMX0gLSAxKX0gKGxfe1xcbGVuKGwpIC0gMX0pX2pdIl0sWzIsMSwiXFxzdW1fe2kgPSAwfV57XFxsZW4obCkgLSAxfSBsX2kiXSxbMCwwLCJcXExpc3ReMyBBIl0sWzAsMSwiXFxMaXN0XjIgQSJdLFsxLDAsIlxcTGlzdF4yIEEiXSxbMSwxLCJcXExpc3QgQSJdLFszLDEsIlxcc3VtX3tpLGp9IChsX3tpfSlqIl0sWzAsMSwiIiwwLHsic3R5bGUiOnsidGFpbCI6eyJuYW1lIjoibWFwcyB0byJ9fX1dLFswLDIsIiIsMix7InN0eWxlIjp7InRhaWwiOnsibmFtZSI6Im1hcHMgdG8ifX19XSxbMyw1LCJcXExpc3QgXFxtdV9BIl0sWzUsNiwiXFxtdV9BIl0sWzMsNCwiXFxtdV97XFxMaXN0IEF9IiwyXSxbNCw2LCJcXG11X0EiLDJdLFsxLDcsIiIsMCx7InN0eWxlIjp7InRhaWwiOnsibmFtZSI6Im1hcHMgdG8ifX19XSxbMiw3LCIiLDIseyJzdHlsZSI6eyJ0YWlsIjp7Im5hbWUiOiJtYXBzIHRvIn19fV1d
    \[\begin{tikzcd}
    	{\List^3 A} & {\List^2 A} & l & {[ \sum_{j = 0}^{\len(l_0) - 1} (l_0)_j, ...,  \sum_{j=0}^{\len(l_{\len(l) - 1} - 1)} (l_{\len(l) - 1})_j]} \\
    	{\List^2 A} & {\List A} & {\sum_{i = 0}^{\len(l) - 1} l_i} & {\sum_{i,j} (l_{i})_j}
    	\arrow["{\List \mu_A}", from=1-1, to=1-2]
    	\arrow["{\mu_{\List A}}"', from=1-1, to=2-1]
    	\arrow["{\mu_A}", from=1-2, to=2-2]
    	\arrow[maps to, from=1-3, to=1-4]
    	\arrow[maps to, from=1-3, to=2-3]
    	\arrow[maps to, from=1-4, to=2-4]
    	\arrow["{\mu_A}"', from=2-1, to=2-2]
    	\arrow[maps to, from=2-3, to=2-4]
    \end{tikzcd}
    \text{.}
    \]

    \paragraph{Kleisli category $\Kl(\List)$.}     
    \hfill \break
    One of the possible interpretations of $f: A \to \List B$  is a non-deterministic calculation: $a$ is mapped by $f$ to the list of all possible results. The composition corresponds to getting all possible results from sequential non-deterministic calculations:  
    % https://q.uiver.app/#q=WzAsNyxbMSwwLCJCIl0sWzEsMSwiQyJdLFswLDAsIkEiXSxbMywwLCJbYl8wLCAuLi4sIGJfbl0iXSxbNCwwLCJhIl0sWzMsMSwiW1tiXzBeMCxiXzFeMCwuLi4sYl97bV8wfV4wXSAsIC4uLiBbYl8wXm4sYl8xXm4sLi4uLGJfe21fbn1ebl1dIl0sWzQsMSwiW2JfMF4wLGJfMV4wLC4uLixiX3ttXzB9XjAsIC4uLiwgYl8wXm4sYl8xXm4sLi4uLGJfe21fbn1ebl0iXSxbMiwwLCJmXlxcYnVsbGV0Il0sWzAsMSwiZ15cXGJ1bGxldCJdLFsyLDEsImdeXFxidWxsZXQgXFxjaXJjIGZeXFxidWxsZXQiLDJdLFs0LDMsImYiLDIseyJzdHlsZSI6eyJ0YWlsIjp7Im5hbWUiOiJtYXBzIHRvIn19fV0sWzMsNSwiVGcgPSBbZ18wKGJfMCksIC4uLiwgZ19uKGJfbildIiwyLHsic3R5bGUiOnsidGFpbCI6eyJuYW1lIjoibWFwcyB0byJ9fX1dLFs1LDYsIlxcbXVfQiA9IFxcc3VtX2kgZ19pKGJfaSkiLDIseyJvZmZzZXQiOjEsInN0eWxlIjp7InRhaWwiOnsibmFtZSI6Im1hcHMgdG8ifX19XSxbNCw2LCIiLDAseyJzdHlsZSI6eyJ0YWlsIjp7Im5hbWUiOiJtYXBzIHRvIn19fV1d
    \[\begin{tikzcd}
    	A & B && {[b_0, ..., b_n]} & a \\
    	& C && {[[b_0^0,b_1^0,...,b_{m_0}^0] , ... [b_0^n,b_1^n,...,b_{m_n}^n]]} & {[b_0^0,b_1^0,...,b_{m_0}^0, ..., b_0^n,b_1^n,...,b_{m_n}^n]}
    	\arrow["{f^\bullet}", from=1-1, to=1-2]
    	\arrow["{g^\bullet \circ f^\bullet}"', from=1-1, to=2-2]
    	\arrow["{g^\bullet}", from=1-2, to=2-2]
    	\arrow["{Tg = [g_0(b_0), ..., g_n(b_n)]}"', maps to, from=1-4, to=2-4]
    	\arrow["f"', maps to, from=1-5, to=1-4]
    	\arrow[maps to, from=1-5, to=2-5]
    	\arrow["{\mu_B = \sum_i g_i(b_i)}"', shift right, maps to, from=2-4, to=2-5]
    \end{tikzcd}
    \text{.}
    \]

\newpage

\subsubsection{Exception monad}
    \paragraph{Definition and monad structure.}  
    \hfill \break
    Consider a type $E$ of exceptions (error values) and a functor $\Exc: \mc{S}et \to \mc{S}et$, which sends a type to its coproduct with the exception type:
    % https://q.uiver.app/#q=WzAsNCxbMSwwLCJBICsgRSJdLFswLDAsIkEiXSxbMCwxLCJCIl0sWzEsMSwiQiArIEUiXSxbMSwyLCJmIiwyXSxbMCwzLCJmICsgXFxpZCJdXQ==
    \[\begin{tikzcd}
    	A & {A + E} \\
    	B & {B + E}
    	\arrow["f"', from=1-1, to=2-1]
    	\arrow["{f + \id}", from=1-2, to=2-2]
    \end{tikzcd}
    \text{.}
    \]
    The monad structure
    $$
    (\Exc, \eta: \id \to \Exc, \mu: \Exc^2 \to \Exc).
    $$
    is defined as follows:
    \begin{itemize}
        \item[$\bullet$] Unit represents successful calculation: $$\eta_A := i_1: A \to A + E,$$
        \item[$\bullet$] Multiplication represents the error propagation:
        $$\mu_A: A + E + E \to A + E = \id_A + \nabla_{E, E} = \nabla_{A + E, E},$$
        where $\nabla$ is the codiagonal map. 
    \end{itemize}
    The diagrams are obvious on $\mc{S}$et:
    % https://q.uiver.app/#q=WzAsNixbMywwLCIoQSArIEUpICsgRSJdLFswLDAsIkEgKyBFIl0sWzMsMSwiQSArIEUiXSxbNCwwLCJhX0EgXFwgfHwgXFwgXFx0ZXh0cm17ZXJyb3J9X0UiXSxbNSwwLCIoYSBcXCB8fCBcXCBcXHRleHRybXtlcnJvcn0pX3tBICsgRX0iXSxbNSwxLCJhX0EgXFwgfHwgXFwgXFx0ZXh0cm17ZXJyb3J9X0UiXSxbMCwyLCJcXG5hYmxhX3tBICsgRSwgRX0iXSxbMSwyLCJcXGlkIiwyXSxbMSwwLCJcXGV0YV97QSArIEV9ID0gaV97MSwgQSArIEV9Il0sWzMsNCwiIiwwLHsic3R5bGUiOnsidGFpbCI6eyJuYW1lIjoibWFwcyB0byJ9fX1dLFs0LDUsIiIsMCx7InN0eWxlIjp7InRhaWwiOnsibmFtZSI6Im1hcHMgdG8ifX19XSxbMyw1LCIiLDIseyJzdHlsZSI6eyJ0YWlsIjp7Im5hbWUiOiJtYXBzIHRvIn19fV1d
    \[\begin{tikzcd}
    	{A + E} &&& {(A + E) + E} & {a_A \ || \ \textrm{error}_E} & {(a \ || \ \textrm{error})_{A + E}} \\
    	&&& {A + E} && {a_A \ || \ \textrm{error}_E}
    	\arrow["{\eta_{A + E} = i_{1, A + E}}", from=1-1, to=1-4]
    	\arrow["\id"', from=1-1, to=2-4]
    	\arrow["{\nabla_{A + E, E}}", from=1-4, to=2-4]
    	\arrow[maps to, from=1-5, to=1-6]
    	\arrow[maps to, from=1-5, to=2-6]
    	\arrow[maps to, from=1-6, to=2-6]
    \end{tikzcd}    
    \text{,}
    \]
    % https://q.uiver.app/#q=WzAsNixbMCwwLCIoQSArIEUpICsgRSJdLFszLDAsIkEgKyBFIl0sWzAsMSwiQSArIEUiXSxbNSwwLCJhX0EgXFwgfHwgXFwgXFx0ZXh0cm17ZXJyb3J9X0UiXSxbNCwwLCJhX3tBICsgRX0gXFwgfHwgXFwgXFx0ZXh0cm17ZXJyb3J9X0UiXSxbNCwxLCJhX0EgXFwgfHwgXFwgXFx0ZXh0cm17ZXJyb3J9X0UiXSxbMSwyLCJcXGlkIl0sWzAsMiwiXFxuYWJsYV97QSArIEUsIEV9IiwyXSxbMSwwLCJcXGV0YV9BICsgXFxpZF9FID0gaV97MSwgQX0gKyBcXGlkX0UiLDJdLFs0LDUsIiIsMix7InN0eWxlIjp7InRhaWwiOnsibmFtZSI6Im1hcHMgdG8ifX19XSxbMyw0LCIiLDIseyJzdHlsZSI6eyJ0YWlsIjp7Im5hbWUiOiJtYXBzIHRvIn19fV0sWzMsNSwiIiwwLHsic3R5bGUiOnsidGFpbCI6eyJuYW1lIjoibWFwcyB0byJ9fX1dXQ==
    \[\begin{tikzcd}
    	{(A + E) + E} &&& {A + E} & {a_{A + E} \ || \ \textrm{error}_E} & {a_A \ || \ \textrm{error}_E} \\
    	{A + E} &&&& {a_A \ || \ \textrm{error}_E}
    	\arrow["{\nabla_{A + E, E}}"', from=1-1, to=2-1]
    	\arrow["{\eta_A + \id_E = i_{1, A} + \id_E}"', from=1-4, to=1-1]
    	\arrow["\id", from=1-4, to=2-1]
    	\arrow[maps to, from=1-5, to=2-5]
    	\arrow[maps to, from=1-6, to=1-5]
    	\arrow[maps to, from=1-6, to=2-5]
    \end{tikzcd}
    \text{,}
    \]  
    % https://q.uiver.app/#q=WzAsOCxbMCwwLCIoKEEgKyBFKSArIEUgKSArIEUiXSxbMiwwLCIoQSArIEUpICsgRSJdLFswLDIsIihBICsgRSkgKyBFIl0sWzIsMiwiQStFIl0sWzMsMCwiYV9BIFxcIHx8IFxcICBcXHRleHRybXtlcnJvcn1fe0VeeygxKX0rRV57KDIpfStFXnsoMyl9fSJdLFszLDIsImFfQSAgXFwgfHwgXFwgXFx0ZXh0cm17ZXJyb3J9X3tFXnsoMSl9ICsgRV57KDIpfX0iXSxbNCwwLCJhX0EgXFwgfHwgXFwgIFxcdGV4dHJte2Vycm9yfV97RV57KDIpfStFXnsoMyl9fSJdLFs0LDIsImFfQSAgXFwgfHwgXFwgXFx0ZXh0cm17ZXJyb3J9X0UiXSxbMCwxLCJcXG5hYmxhX3tBICsgRSwgRX0gKyBcXGlkX0UiXSxbMCwyLCJcXG5hYmxhX3soQSArIEUpICsgRSwgRX0iLDJdLFsyLDMsIlxcbmFibGFfe0ErRSwgRX0iLDJdLFsxLDMsIlxcbmFibGFfe0ErRSwgRX0iXSxbNCw1LCIiLDAseyJzdHlsZSI6eyJ0YWlsIjp7Im5hbWUiOiJtYXBzIHRvIn19fV0sWzQsNiwiIiwyLHsic3R5bGUiOnsidGFpbCI6eyJuYW1lIjoibWFwcyB0byJ9fX1dLFs1LDcsIiIsMCx7InN0eWxlIjp7InRhaWwiOnsibmFtZSI6Im1hcHMgdG8ifX19XSxbNiw3LCIiLDIseyJzdHlsZSI6eyJ0YWlsIjp7Im5hbWUiOiJtYXBzIHRvIn19fV1d
    \[\begin{tikzcd}[column sep=scriptsize]
    	{((A + E) + E ) + E} && {(A + E) + E} & {a_A \ || \  \textrm{error}_{E^{(1)}+E^{(2)}+E^{(3)}}} & {a_A \ || \  \textrm{error}_{E^{(2)}+E^{(3)}}} \\
    	\\
    	{(A + E) + E} && {A+E} & {a_A  \ || \ \textrm{error}_{E^{(1)} + E^{(2)}}} & {a_A  \ || \ \textrm{error}_E}
    	\arrow["{\nabla_{A + E, E} + \id_E}", from=1-1, to=1-3]
    	\arrow["{\nabla_{(A + E) + E, E}}"', from=1-1, to=3-1]
    	\arrow["{\nabla_{A+E, E}}", from=1-3, to=3-3]
    	\arrow[maps to, from=1-4, to=1-5]
    	\arrow[maps to, from=1-4, to=3-4]
    	\arrow[maps to, from=1-5, to=3-5]
    	\arrow["{\nabla_{A+E, E}}"', from=3-1, to=3-3]
    	\arrow[maps to, from=3-4, to=3-5]
    \end{tikzcd}
    \text{.}
    \] 
    Here $a_A \ || \ b_B $ informally denotes the element of $A + B$, subscripts annotate the domains.

    \paragraph{Kleisli category $\Kl(\Exc)$.}     
    \hfill \break
    The Kleisli composition for the Exception monad tracks the source of the possible error:
    % https://q.uiver.app/#q=WzAsNyxbMCwwLCJBIl0sWzEsMCwiQiJdLFsxLDEsIkMiXSxbNCwwLCJhX0EiXSxbMywwLCJmKGEpX0IgXFwgfHwgXFwgXFx0ZXh0cm17ZXJyb3J9X0UiXSxbMywxLCIoZ2YoYSlfQyAgXFwgfHwgXFwgXFx0ZXh0cm17ZXJyb3J9KV97QyArIEV9IFxcIHx8XFwgIFxcdGV4dHJte2Vycm9yfV9FIl0sWzQsMSwiZ2YoYSlfQyAgXFwgfHxcXCAgIFxcdGV4dHJte2Vycm9yfV9FIl0sWzAsMSwiZl5cXGJ1bGxldCJdLFsxLDIsImdeXFxidWxsZXQiXSxbMCwyLCJnXlxcYnVsbGV0IFxcY2lyYyBmXlxcYnVsbGV0IiwyXSxbMyw0LCJmIiwyLHsic3R5bGUiOnsidGFpbCI6eyJuYW1lIjoibWFwcyB0byJ9fX1dLFs0LDUsImcgKyBcXGlkX0UiLDIseyJzdHlsZSI6eyJ0YWlsIjp7Im5hbWUiOiJtYXBzIHRvIn19fV0sWzUsNiwiXFxtdV9DIiwyLHsic3R5bGUiOnsidGFpbCI6eyJuYW1lIjoibWFwcyB0byJ9fX1dLFszLDYsIiIsMCx7InN0eWxlIjp7InRhaWwiOnsibmFtZSI6Im1hcHMgdG8ifX19XV0=
    \[\begin{tikzcd}
    	A & B && {f(a)_B \ || \ \textrm{error}_E} & {a_A} \\
    	& C && {(gf(a)_C  \ || \ \textrm{error})_{C + E} \ ||\  \textrm{error}_E} & {gf(a)_C  \ ||\   \textrm{error}_E}
    	\arrow["{f^\bullet}", from=1-1, to=1-2]
    	\arrow["{g^\bullet \circ f^\bullet}"', from=1-1, to=2-2]
    	\arrow["{g^\bullet}", from=1-2, to=2-2]
    	\arrow["{g + \id_E}"', maps to, from=1-4, to=2-4]
    	\arrow["f"', maps to, from=1-5, to=1-4]
    	\arrow[maps to, from=1-5, to=2-5]
    	\arrow["{\mu_C}"', maps to, from=2-4, to=2-5]
    \end{tikzcd}
    \text{.}
    \]
\newpage

\section{Comonads (not started)}



\end{document}
