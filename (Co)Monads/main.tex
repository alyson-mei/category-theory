\documentclass[12pt]{article}
% Page setup
\usepackage[margin=1in]{geometry}
\usepackage{parskip}
\usepackage{titling}
\setlength{\droptitle}{-6em} 

% Math packages
\usepackage{amsmath, amsthm, amssymb, amsfonts}
\usepackage{mathtools}
\makeatletter
\g@addto@macro\th@plain{\normalfont} % remove italics from theorems (plain style)
\makeatother

% Commutative diagrams
\usepackage{tikz-cd}

% Theorem environments
\newtheorem{theorem}{Theorem}[section]
\newtheorem{lemma}[theorem]{Lemma}
\newtheorem{proposition}[theorem]{Proposition}
\newtheorem{corollary}[theorem]{Corollary}
\newtheorem{fact}[theorem]{Fact}
\newtheorem{remark}[theorem]{Remark}

\theoremstyle{definition}
\newtheorem{definition}[theorem]{Definition}
\newtheorem{example}[theorem]{Example}

% Useful commands for category theory
\newcommand{\mc}{\mathcal}
\newcommand{\ini}{\operatorname{ini}}
\newcommand{\ter}{\operatorname{ter}}
\newcommand{\List}{\operatorname{List}}
\newcommand{\lop}{\operatorname{[ \ ]}}
\newcommand{\len}{\operatorname{len}}
\newcommand{\plus}{\operatorname{+}}

\newcommand{\Cat}{\mathcal{C}at}
\newcommand{\Vect}{\mathrm{Vect}_\mathbf{k}}
\newcommand{\Hom}{\operatorname{Hom}}
\newcommand{\Nat}{\operatorname{Nat}}
\newcommand{\id}{\mathrm{id}}
\newcommand{\op}{\mathrm{op}}
\newcommand{\obj}{\operatorname{obj}}
\newcommand{\arr}{\operatorname{arr}}
\newcommand{\adj}{\operatorname{adj}}
\newcommand{\coadj}{\operatorname{coadj}}
\newcommand{\colim}{\operatorname*{colim}}

% Arrows
\newcommand{\ra}{\rightarrow}
\newcommand{\from}{\leftarrow}
\newcommand{\To}{\Rightarrow}
\newcommand{\xto}{\xrightarrow}
\newcommand{\xfrom}{\xleftarrow}

\renewcommand{\labelitemi}{--} % Level 1: Dashes
\renewcommand{\labelitemii}{$\circ$} % Level 2: Circles
\renewcommand{\labelitemiii}{$\bullet$} % Level 3: Bullets (default)


\title{(Co)Monads}
\author{Alyson Mei}
\date{\today}

\begin{document}

\maketitle

\section{Introduction}

\paragraph{Conventions.}
Throughout this text, we adopt the following conventions:
\begin{itemize}
    \item All categories are assumed to be locally small.
    \item Standard polymorphic notation is used freely when no ambiguity arises.
    \item All diagrams are assumed to commute unless stated otherwise.
\end{itemize}




\section{Monads}

\begin{definition}[Monoid object]

Let $(\mc{C}, \otimes, I)$ be a monoidal category. A monoid object in $\mc{C}$ consists of:
\begin{itemize}
    \item[$\bullet$] an object $M \in \obj(\mc{C})$,
    \item[$\bullet$] a multiplication morphism
    \[
        \mu : M \otimes M \to M,
    \]
    \item[$\bullet$] a unit morphism
    \[
        \eta : I \to M,
    \]
\end{itemize}
satisfying the unit and associativity  diagrams:
% https://q.uiver.app/#q=WzAsOCxbMSwwLCJNIFxcdGltZXMgTSJdLFswLDAsIk0iXSxbMSwxLCJNIl0sWzIsMCwiTSJdLFs0LDAsIk1eMyJdLFs0LDEsIk1eMiJdLFs1LDAsIk1eMiJdLFs1LDEsIk0iXSxbMSwwLCJNIFxcdGltZXMgXFxldGEiXSxbMCwyLCJcXG11IiwxXSxbMSwyLCJcXGlkIiwyXSxbMywyLCJcXGlkIl0sWzMsMCwiXFxldGEgXFx0aW1lcyBNIiwyXSxbNCw1LCIgTSBcXHRpbWVzIFxcbXUiLDJdLFs0LDYsIlxcbXUgXFx0aW1lcyBNIl0sWzYsNywiXFxtdSJdLFs1LDcsIlxcbXUiLDJdXQ==
\[\begin{tikzcd}
	M & {M \otimes M} & M && {M^{\otimes 3}} & {M^{\otimes2}} \\
	& M &&& {M^{\otimes 2}} & M
	\arrow["{M \otimes \eta}", from=1-1, to=1-2]
	\arrow["\id"', from=1-1, to=2-2]
	\arrow["\mu"{description}, from=1-2, to=2-2]
	\arrow["{\eta \otimes M}"', from=1-3, to=1-2]
	\arrow["\id", from=1-3, to=2-2]
	\arrow["{\mu \otimes M}", from=1-5, to=1-6]
	\arrow["{ M \otimes \mu}"', from=1-5, to=2-5]
	\arrow["\mu", from=1-6, to=2-6]
	\arrow["\mu"', from=2-5, to=2-6]
\end{tikzcd}
\text{.}
\]
\end{definition}

\begin{example}
\hfill
\begin{itemize}
    \item Categories with finite products $(\mc{C}, \times, 1)$ are monoidal:
    \begin{itemize}
        \item In $\mc{S}et$, monoid objects are ordinary monoids;
        \item In $\Vect$, monoid objects are associative unital algebras over a field $\mathbf{k}$;
    \end{itemize}
    \item The category of endofunctors $([\mc{C}, \mc{C}], \circ, \id)$ is monoidal; monoid objects in this category are called monads.
\end{itemize}
\end{example}



\begin{definition}[Monad]
Monad is a triple $$(M: \mc{C} \to \mc{C}, \eta: \id_\mc{C} \to M, \mu: M^2 \to M),$$
satisfying the unit and associativity diagrams:
% https://q.uiver.app/#q=WzAsOCxbMSwwLCJNXjIiXSxbMiwwLCJNIl0sWzEsMSwiTSJdLFswLDAsIk0iXSxbMywwLCJNXjMiXSxbMywxLCJNXjIiXSxbNCwwLCJNXjIiXSxbNCwxLCJNIl0sWzMsMiwiXFxpZF9NIiwyXSxbMSwyLCJcXGlkX00iXSxbMCwyLCJcXG11Il0sWzMsMCwiXFxldGEgXFxjaXJjIE0iXSxbMSwwLCJNIFxcY2lyYyBcXGV0YSIsMl0sWzYsNywiXFxtdSJdLFs0LDUsIlxcbXUgXFxjaXJjIE0iLDJdLFs1LDcsIlxcbXUiLDJdLFs0LDYsIk0gXFxjaXJjIFxcbXUiXV0=
\[\begin{tikzcd}
	M & {M^2} & M & {M^3} & {M^2} \\
	& M && {M^2} & M
	\arrow["{\eta \circ M}", from=1-1, to=1-2]
	\arrow["{\id_M}"', from=1-1, to=2-2]
	\arrow["\mu", from=1-2, to=2-2]
	\arrow["{M \circ \eta}"', from=1-3, to=1-2]
	\arrow["{\id_M}", from=1-3, to=2-2]
	\arrow["{M \circ \mu}", from=1-4, to=1-5]
	\arrow["{\mu \circ M}"', from=1-4, to=2-4]
	\arrow["\mu", from=1-5, to=2-5]
	\arrow["\mu"', from=2-4, to=2-5]
\end{tikzcd}
\text{.}
\]
\end{definition}

\begin{example}[List monad]
Consider the functor $\List : \mc{S}et \to \mc{S}et$, which sends a set to the set of all finite lists of its elements:
% https://q.uiver.app/#q=WzAsNixbMiwwLCJcXExpc3QgIEEiXSxbMiwxLCJcXExpc3QgIEIiXSxbMCwwLCJBIl0sWzAsMSwiQiJdLFszLDAsIlthLCBiLCBjXSJdLFszLDEsIltmKGEpLCBmKGIpLCBmKGMpXSJdLFsyLDMsImYiXSxbMCwxLCJcXExpc3QgZiJdLFs0LDUsIlxcTGlzdCBmIiwwLHsic3R5bGUiOnsidGFpbCI6eyJuYW1lIjoibWFwcyB0byJ9fX1dXQ==
\[\begin{tikzcd}
	A && {\List  A} & {[a_1, a_2, ... a_n]} \\
	B && {\List  B} & {[f(a_1), f(a_2), ..., f(a_n)]}
	\arrow["f", from=1-1, to=2-1]
	\arrow["{\List f}", from=1-3, to=2-3]
	\arrow["{\List f}", maps to, from=1-4, to=2-4]
\end{tikzcd}
\text{.}
\]

From a categorical perspective, the functor $\List$ can be presented as follows:
% https://q.uiver.app/#q=WzAsOCxbMCwwLCJcXG1je1N9ZXQiXSxbMSwwLCJcXG1je1N9ZXReXFxtYXRoYmJ7Tn0iXSxbMSwxLCJcXG1je1N9ZXReXFxtYXRoYmJ7Tn0iXSxbMCwxLCJcXG1je1N9ZXQiXSxbMywwLCJBIl0sWzQsMCwiQV5cXG1hdGhiYntOfSJdLFs0LDEsIihBXm4pX3tuIFxcaW4gXFxtYXRoYmJ7Tn19Il0sWzMsMSwiXFxjb3Byb2Rfe24gXFxpbiBcXG1hdGhiYntOfX0gQV5uIl0sWzAsMSwiXFxEZWx0YSJdLFswLDMsIlxcTGlzdCIsMl0sWzIsMywiXFxjb3Byb2Rfe24gXFxpbiBcXG1hdGhiYntOfX0gIl0sWzEsMiwiXFxwcm9kX3tuIFxcaW4gXFxtYXRoYmJ7Tn19XFx0aW1lc15uIl0sWzUsNiwiIiwwLHsic3R5bGUiOnsidGFpbCI6eyJuYW1lIjoibWFwcyB0byJ9fX1dLFs0LDUsIiIsMCx7InN0eWxlIjp7InRhaWwiOnsibmFtZSI6Im1hcHMgdG8ifX19XSxbNiw3LCIiLDAseyJzdHlsZSI6eyJ0YWlsIjp7Im5hbWUiOiJtYXBzIHRvIn19fV0sWzQsNywiXFxMaXN0IChBKSIsMix7InN0eWxlIjp7InRhaWwiOnsibmFtZSI6Im1hcHMgdG8ifX19XV0=
\[\begin{tikzcd}
	{\mc{S}et} & {\mc{S}et^\mathbb{N}} && A & {A^\mathbb{N}} \\
	{\mc{S}et} & {\mc{S}et^\mathbb{N}} && {\coprod_{n \in \mathbb{N}} A^n} & {(A^n)_{n \in \mathbb{N}}}
	\arrow["\Delta", from=1-1, to=1-2]
	\arrow["\List"', from=1-1, to=2-1]
	\arrow["{\prod_{n \in \mathbb{N}}\times^n}", from=1-2, to=2-2]
	\arrow[maps to, from=1-4, to=1-5]
	\arrow["{\List (A)}"', maps to, from=1-4, to=2-4]
	\arrow[maps to, from=1-5, to=2-5]
	\arrow["{\coprod_{n \in \mathbb{N}} }", from=2-2, to=2-1]
	\arrow[maps to, from=2-5, to=2-4]
\end{tikzcd}
\text{.}\]

//TODO: Specify the $\Delta$ framework

This presentation gives rise to the following API:

\begin{itemize}
    \item The collection of constructors
    $$
    \lop_{n, A} :=  A^n \xto{i_n} \List A := \lambda a_1 \dots a_n. [a_1, \dots, a_n],
    $$
    in particular:
    \begin{itemize}
        \item The single-element list constructor
        $$
        \eta_A : A \to \List A := \lambda a. [a],
        $$
        \item The empty list constructor 
        $$
        \lop_0 : 1 \to \List A := \lambda a. [];
        $$
    \end{itemize}
    \item Length of a list via the power of $A$ in the coproduct:
    $$
    \len_A : \List A \to \mathbf{N};
    $$
    \item List concatenation via the natural isomorphism $A^n \times A^m \cong A^{n+m}$:
% https://q.uiver.app/#q=WzAsOCxbMiwwLCJBXm0gXFx0aW1lcyBBXm4iXSxbMywwLCJcXExpc3QgQSBcXHRpbWVzIFxcTGlzdCBBIl0sWzIsMSwiQV57bSArIG59Il0sWzMsMSwiXFxMaXN0IEEiXSxbMCwwLCJcXGlkIFxcdGltZXMgXFxpZCJdLFsxLDAsIiBcXExpc3RcXHRpbWVzIFxcTGlzdCJdLFswLDEsIlxcaWQiXSxbMSwxLCJcXExpc3QiXSxbMCwxLCJpX20gXFx0aW1lcyBpX24iXSxbMCwyLCJcXGNvbmciLDJdLFsyLDMsImlfe20gKyBufSIsMl0sWzEsMywiK19BIl0sWzYsNywiaSIsMl0sWzUsNywiKyJdLFs0LDYsIlxcc2ltZXEiLDJdLFs0LDUsImkgXFx0aW1lcyBpIl1d
\[\begin{tikzcd}
	{\id \times \id} & { \List\times \List} & {A^m \times A^n} & {\List A \times \List A} \\
	\id & \List & {A^{m + n}} & {\List A}
	\arrow["{i \times i}", from=1-1, to=1-2]
	\arrow["\simeq"', from=1-1, to=2-1]
	\arrow["{+}", from=1-2, to=2-2]
	\arrow["{i_m \times i_n}", from=1-3, to=1-4]
	\arrow["\cong"', from=1-3, to=2-3]
	\arrow["{+_A}", from=1-4, to=2-4]
	\arrow["i"', from=2-1, to=2-2]
	\arrow["{i_{m + n}}"', from=2-3, to=2-4]
\end{tikzcd}
\text{,}
\]
$$+_A: \List A \times \List A \to \List A := \lambda l m. [l_0, ... l_{\len(l) - 1},   m_0, ..., m_{\len(m)  - 1}],$$
$$+_A : (\List A)^n \to \List A := \lambda l_0...l_n. l_0 + ... + l_n;$$
\item List destructors: 
$$\mu_A: \List(\List A) \to \List(A) := \lambda l. \sum_{i = 0}^{\len(l) - 1} l_i,$$
$$\mu_A : \List^nA \to \List A := \mu_A^{\circ n}.$$
\end{itemize}

%Destructor is a left inverse to constructor. Using a polymorphic notation we can write:
%$$  \plus \circ \lop =  \id.$$

First of all, this API provides an internal monoid structure on lists: for a fixed set $A$, we have a monoid object
$$
(\List A, [], +_A)
$$
in $\mc{S}et$. The corresponding diagrams are straightforward in this case.

Second, it provides a monad structure on $\List$:
$$
(\List, \eta: \id \to \List, \mu: \List^2 \to \List).
$$

The corresponding diagrams:
% https://q.uiver.app/#q=WzAsNixbMCwwLCJcXExpc3QgQSJdLFsyLDAsIlxcTGlzdCAoXFxMaXN0IEEpIl0sWzIsMSwiXFxMaXN0IEEiXSxbMywwLCJsIl0sWzQsMCwiW2xdIl0sWzQsMSwibCA9IFtsXV8wIl0sWzAsMSwiXFxldGFfe1xcTGlzdCBBfSJdLFsxLDIsIlxcbXVfQSJdLFswLDIsIlxcaWQiLDJdLFs0LDUsIiIsMCx7InN0eWxlIjp7InRhaWwiOnsibmFtZSI6Im1hcHMgdG8ifX19XSxbMyw0LCIiLDAseyJzdHlsZSI6eyJ0YWlsIjp7Im5hbWUiOiJtYXBzIHRvIn19fV0sWzMsNSwiIiwyLHsic3R5bGUiOnsidGFpbCI6eyJuYW1lIjoibWFwcyB0byJ9fX1dXQ==
\[\begin{tikzcd}
	{\List A} && {\List (\List A)} & l & {[l]} \\
	&& {\List A} && {l = [l]_0}
	\arrow["{\eta_{\List A}}", from=1-1, to=1-3]
	\arrow["\id"', from=1-1, to=2-3]
	\arrow["{\mu_A}", from=1-3, to=2-3]
	\arrow[maps to, from=1-4, to=1-5]
	\arrow[maps to, from=1-4, to=2-5]
	\arrow[maps to, from=1-5, to=2-5]
\end{tikzcd}
\text{,}\]
% https://q.uiver.app/#q=WzAsNixbMCwwLCJcXExpc3QgKFxcTGlzdCBBKSJdLFsyLDAsIlxcTGlzdCBBIl0sWzAsMSwiXFxMaXN0IEEiXSxbNCwwLCJsIl0sWzMsMCwiW1tsXzBdLCAuLi4sIFtsX3tcXGxlbihsKSAtIDF9XV0iXSxbMywxLCJsID0gXFxzdW1fe2k9MH1ee1xcbGVuKGwpIC0gMX0gW2xfaV0iXSxbMSwyLCJcXGlkIl0sWzEsMCwiXFxMaXN0IFxcZXRhX0EiLDJdLFswLDIsIlxcbXVfQSIsMl0sWzQsNSwiIiwyLHsic3R5bGUiOnsidGFpbCI6eyJuYW1lIjoibWFwcyB0byJ9fX1dLFszLDQsIiIsMix7InN0eWxlIjp7InRhaWwiOnsibmFtZSI6Im1hcHMgdG8ifX19XSxbMyw1LCIiLDAseyJzdHlsZSI6eyJ0YWlsIjp7Im5hbWUiOiJtYXBzIHRvIn19fV1d
\[\begin{tikzcd}
	{\List (\List A)} && {\List A} & {[[l_0], ..., [l_{\len(l) - 1}]]} & l \\
	{\List A} &&& {l = \sum_{i=0}^{\len(l) - 1} [l_i]}
	\arrow["{\mu_A}"', from=1-1, to=2-1]
	\arrow["{\List \eta_A}"', from=1-3, to=1-1]
	\arrow["\id", from=1-3, to=2-1]
	\arrow[maps to, from=1-4, to=2-4]
	\arrow[maps to, from=1-5, to=1-4]
	\arrow[maps to, from=1-5, to=2-4]
\end{tikzcd}
\text{,}\]
% https://q.uiver.app/#q=WzAsOCxbMiwwLCJsIl0sWzMsMCwiWyBcXHN1bV97aiA9IDB9XntcXGxlbihsXzApIC0gMX0gKGxfMClfaiwgLi4uLCAgXFxzdW1fe2o9MH1ee1xcbGVuKGxfe1xcbGVuKGwpIC0gMX0gLSAxKX0gKGxfe1xcbGVuKGwpIC0gMX0pX2pdIl0sWzIsMSwiXFxzdW1fe2kgPSAwfV57XFxsZW4obCkgLSAxfSBsX2kiXSxbMCwwLCJcXExpc3ReMyBBIl0sWzAsMSwiXFxMaXN0XjIgQSJdLFsxLDAsIlxcTGlzdF4yIEEiXSxbMSwxLCJcXExpc3QgQSJdLFszLDEsIlxcc3VtX3tpLGp9IChsX3tpfSlqIl0sWzAsMSwiIiwwLHsic3R5bGUiOnsidGFpbCI6eyJuYW1lIjoibWFwcyB0byJ9fX1dLFswLDIsIiIsMix7InN0eWxlIjp7InRhaWwiOnsibmFtZSI6Im1hcHMgdG8ifX19XSxbMyw1LCJcXExpc3QgXFxtdV9BIl0sWzUsNiwiXFxtdV9BIl0sWzMsNCwiXFxtdV97XFxMaXN0IEF9IiwyXSxbNCw2LCJcXG11X0EiLDJdLFsxLDcsIiIsMCx7InN0eWxlIjp7InRhaWwiOnsibmFtZSI6Im1hcHMgdG8ifX19XSxbMiw3LCIiLDIseyJzdHlsZSI6eyJ0YWlsIjp7Im5hbWUiOiJtYXBzIHRvIn19fV1d
\[\begin{tikzcd}
	{\List^3 A} & {\List^2 A} & l & {[ \sum_{j = 0}^{\len(l_0) - 1} (l_0)_j, ...,  \sum_{j=0}^{\len(l_{\len(l) - 1} - 1)} (l_{\len(l) - 1})_j]} \\
	{\List^2 A} & {\List A} & {\sum_{i = 0}^{\len(l) - 1} l_i} & {\sum_{i,j} (l_{i})_j}
	\arrow["{\List \mu_A}", from=1-1, to=1-2]
	\arrow["{\mu_{\List A}}"', from=1-1, to=2-1]
	\arrow["{\mu_A}", from=1-2, to=2-2]
	\arrow[maps to, from=1-3, to=1-4]
	\arrow[maps to, from=1-3, to=2-3]
	\arrow[maps to, from=1-4, to=2-4]
	\arrow["{\mu_A}"', from=2-1, to=2-2]
	\arrow[maps to, from=2-3, to=2-4]
\end{tikzcd}
\text{.}
\]
\end{example}

\newpage

\section{Comonads}



\end{document}
